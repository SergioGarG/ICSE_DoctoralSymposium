\documentclass[sigconf,review]{acmart}
\acmConference[ICSE 2018]{40th International Conference on Software Engineering}{May 27--June 3, 2018}{Gothenburg, Sweden}
\acmYear{2018}

\usepackage{booktabs} % For formal tables
\usepackage{centernot}
\usepackage{algorithm}
\usepackage{algorithmic}
\usepackage{amsmath,amssymb,amsfonts}
\usepackage{balance}

\usepackage[xcolor=orange]{changes}
\definechangesauthor[name={Claudio Menghi},color=orange]{CM}
\definechangesauthor[name={Sergio Garcia},color=red]{SG}
%\definechangesauthor[name={Patrizio},color=blue]{PP}
\usepackage[colorinlistoftodos,prependcaption,textsize=tiny]{todonotes}
%\newcommand{\sergio}[1]{\todo[color=blue]{\textsf{SG} #1}}

\usepackage[inline]{enumitem}
\usepackage[multiple]{footmisc}

\setlength{\belowcaptionskip}{-10pt}

\newboolean{showcomments}
\setboolean{showcomments}{true} % toggle to show or hide comments
\ifthenelse{\boolean{showcomments}}
{\newcommand{\nb}[2]{
  \fcolorbox{black}{yellow}{\bfseries\sffamily\scriptsize#1}
  {\sf\small$\blacktriangleright$\textit{#2}$\blacktriangleleft$}
 }
 \newcommand{\version}{\emph{\scriptsize$-$working$-$}}
}
{\newcommand{\nb}[2]{}
 \newcommand{\version}{}
}
\newcommand\patrizio[1]{\nb{Patrizio}{#1}}
\newcommand\claudio[1]{\nb{Claudio}{#1}}
\newcommand\sergio[1]{\nb{Sergio}{#1}}
\newcommand\jana[1]{\nb{Jana}{#1}}

\newcommand{\Sync}{\ensuremath{Meet}}
\newcommand{\cla}[1]{\textcolor{red}{{#1}}}

\newcommand{\powerset}{\raisebox{.15\baselineskip}{\Large\ensuremath{\wp}}}
\newcommand{\robot}{\ensuremath{\T}}

\newcommand{\INPUT}{\textbf{Input} }
\newcommand{\OUTPUT}{\textbf{Output} }

\newcommand{\ra}{$\rightarrow$}
\newcommand{\ugh}[1]{\textcolor{red}{\uwave{#1}}} % please rephrase
\newcommand{\ins}[1]{\textcolor{blue}{\uline{#1}}} % please insert
\newcommand{\del}[1]{\textcolor{red}{\sout{#1}}} % please delete
\newcommand{\chg}[2]{\textcolor{red}{\sout{#1}}{\ra}\textcolor{blue}{\uline{#2}}}

\newcommand{\TSi}{{\T_i=(S_i,\init_{i},{A_i} ,T_i)}}
\newcommand{\N}{\mathcal{H}}
\newcommand{\M}{{M}}
\newcommand{\model}{\mathcal{M}}
\newcommand{\I}{\mathcal{I}}
\newcommand{\D}{{D}}
\renewcommand{\O}{\mathcal{O}}

\newcommand{\APs}{\mathbf{\Pi}}
\newcommand{\Lang}{\mathcal{L}} %language
\newcommand{\Set}{\mathsf{S}} %set
\newcommand{\Spec}{\mathbf{\Phi}}
\newcommand{\Epsilon}{\mathcal{E}}
\renewcommand{\i}{\iota}
\newcommand{\Nat}{\mathbb{N}} %natural numbers
\newcommand{\Real}{\mathbb{R}}
\newcommand{\Next}{\mathsf{X}}
\newcommand{\Until}{\mathsf{U}}
\newcommand{\Always}{\mathsf{G}}
\newcommand{\Event}{\mathsf{F}}
\newcommand{\false}{\mathit{false}}
\newcommand{\true}{\mathit{true}}
\newcommand{\trueval}{\ensuremath{\top}}
\newcommand{\falseval}{\ensuremath{\bot}}
\newcommand{\maybe}{\ensuremath{?}}
\renewcommand{\epsilon}{\varepsilon}
\newcommand{\prop}{\pi}
\newcommand{\ie}{{i.e., }}
\newcommand{\eg}{{e.g., }}
\newcommand{\progressive}{\varpi}
\newcommand{\move}{\mathit{move}}
\newcommand{\h}{h}
\renewcommand{\H}{H}
\newcommand{\parti}{\mathit{P}}
\newcommand{\Alpha}{\mathbf{\Sigma}}
\renewcommand{\mod}{\mathrm{\, mod \, }}
\newcommand{\suc}{\mathit{succ}}
\newcommand{\dist}{\mathrm{dist}}
\newcommand{\proj}{\mathrm{proj}}
\newcommand{\parspace}{\vskip 0.05in}


\makeatletter
\newcommand*{\da@rightarrow}{\mathchar"0\hexnumber@\symAMSa 4B }
\newcommand*{\da@leftarrow}{\mathchar"0\hexnumber@\symAMSa 4C }
\newcommand*{\xdashrightarrow}[2][]{%
  \mathrel{%
    \mathpalette{\da@xarrow{#1}{#2}{}\da@rightarrow{\,}{}}{}%
  }%
}
\newcommand{\xdashleftarrow}[2][]{%
  \mathrel{%
    \mathpalette{\da@xarrow{#1}{#2}\da@leftarrow{}{}{\,}}{}%
  }%
}
\newcommand*{\da@xarrow}[7]{%
  % #1: below
  % #2: above
  % #3: arrow left
  % #4: arrow right
  % #5: space left 
  % #6: space right
  % #7: math style 
  \sbox0{$\ifx#7\scriptstyle\scriptscriptstyle\else\scriptstyle\fi#5#1#6\m@th$}%
  \sbox2{$\ifx#7\scriptstyle\scriptscriptstyle\else\scriptstyle\fi#5#2#6\m@th$}%
  \sbox4{$#7\dabar@\m@th$}%
  \dimen@=\wd0 %
  \ifdim\wd2 >\dimen@
    \dimen@=\wd2 %   
  \fi
  \count@=2 %
  \def\da@bars{\dabar@\dabar@}%
  \@whiledim\count@\wd4<\dimen@\do{%
    \advance\count@\@ne
    \expandafter\def\expandafter\da@bars\expandafter{%
    }%
  }%  
  \mathrel{#3}%
  \mathrel{%   
    \mathop{\da@bars}\limits
    \ifx\\#1\\%
    \else
      _{\copy0}%
    \fi
    \ifx\\#2\\%
    \else
      ^{\copy2}%
    \fi
  }%   
  \mathrel{#4}%
}

\everymath{\vadjust{\nobreak\null}}

\makeatletter
\def\old@comma{,}
\catcode`\,=13
\def,{%
  \ifmmode%
    \old@comma\discretionary{}{}{}%
  \else%
    \old@comma%
  \fi%
}
\makeatother

\def\HiLi{\leavevmode\rlap{\hbox to \hsize{\color{yellow!50}\leaders\hrule height .8\baselineskip depth .5ex\hfill}}}

\newif\ifextended
\extendedfalse
%\extendedtrue

\ifextended
\newcommand{\extended}[1]{\textcolor{red}{#1}} 
\newcommand{\notextended}[1]{}
\else
\newcommand{\extended}[1]{}
\newcommand{\notextended}[1]{#1}
\fi

\begin{document}
\title{Software engineering multi-robot applications}


\author{Sergio Garc\'{i}a}
\email{sergio.garcia@gu.se}
\affiliation{%
  \institution{Chalmers University of Technology $|$ University of Gothenburg, Gothenburg, Sweden}
  %\city{Gothenburg} 
  %\state{Sweden} 
}

\begin{abstract}
The number of robotic applications that are being developed is increasing exponentially both in industry and academia.
However, those applications are not developed through well-defined system development processes.
This fact leads to several time-consuming tasks as the necessity of starting projects from scratch, the inability of reusing already developed packages and of exchanging the already implemented ones.
Besides, robot applications are increasingly based on \emph{teams} of autonomous robots that work collaboratively to accomplish complex missions.
It also increases the complexity of the controlling application.

In this PhD project, we aim to bring software engineering best practices to robotic systems in order to produce processes, architectural models and methods to be used by developers in order to tackle common challenges as reusability, variability and modularity.
The goal is to cut down the waste of time when developing robotic applications.
Furthermore, for our multi-robot applications we study how to manage and exploit emergent behaviours and how to perform a collaborative adaptation based on them.
To validate our results we make use of different models of service robots. %in several case studies.
\end{abstract}

%
% The code below should be generated by the tool at
% http://dl.acm.org/ccs.cfm
% Please copy and paste the code instead of the example below. 
%
%\begin{CCSXML}
%<ccs2012>
% <concept>
%  <concept_id>10010520.10010553.10010562</concept_id>
%  <concept_desc>Computer systems organization~Embedded systems</concept_desc>
%  <concept_significance>500</concept_significance>
% </concept>
% <concept>
%  <concept_id>10010520.10010575.10010755</concept_id>
%  <concept_desc>Computer systems organization~Redundancy</concept_desc>
%  <concept_significance>300</concept_significance>
% </concept>
% <concept>
%  <concept_id>10010520.10010553.10010554</concept_id>
%  <concept_desc>Computer systems organization~Robotics</concept_desc>
%  <concept_significance>100</concept_significance>
% </concept>
% <concept>
%  <concept_id>10003033.10003083.10003095</concept_id>
%  <concept_desc>Networks~Network reliability</concept_desc>
%  <concept_significance>100</concept_significance>
% </concept>
%</ccs2012>  
%\end{CCSXML}
%
%\ccsdesc[500]{Computer systems organization~Embedded systems}
%\ccsdesc[300]{Computer systems organization~Redundancy}
%\ccsdesc{Computer systems organization~Robotics}
%\ccsdesc[100]{Networks~Network reliability}


%\keywords{ACM proceedings, \LaTeX, text tagging}


\maketitle

\section{Introduction}
\label{sec:introduction}
%Helping designers to engineer robotic systems is one of the timely application areas of software engineering. 
%Designing %good 
%robotic systems requires solving %several 
%numerous problems~\cite{ljungblad2005designing}, such as the selection of the types of the robot to be employed (e.g., humanoid, robotic arm), the analysis of the required robot properties (e.g., emergence, emotional), the analysis of the place in which the robot operates (e.g., on the bus, in a birthday party), the activity the robot has to perform (e.g., move object, reach a location), and the users it has to serve (e.g., taxi driver, rock star). 
%These aspects are then used by designers in the selection of appropriate planners.

\toolName provides a planner where a robot application is defined using finite transition systems.
A \emph{planner} is  a software component that receives as input a model of the robotic application and computes  a set of actions (a \emph{plan}) that, if performed, allows the achievement of a desired mission~\cite{latombe2012robot}.
Each robot application contains the robots that conform the team and the mission that they have to achieve.

A \emph{global mission} represents the high-level mission that must be accomplished by the whole team \cite{kloetzer2011multi,loizou2005automated,quottrup2004multi} and that is decomposed into a set of \emph{local missions}\cite{schillinger2016decomposition,guo2015multi,guo2015multi,tumova2016multi}.
Every robot is commanded to achieve a local mission, specified as a LTL property.
As seen in \cite{tumova2016multi}, this collaborative fashion of accomplishing the global mission is performed in a \emph{decentralized} way.
Each robot that is part of a subset of the team computes the solution for its own sub-mission, avoiding the expensive fully centralized planning and making it more robust to local problems.

Nowadays, most of the planners consider the model of the environment as known and not dynamic~\cite{7139412}. 
However, this is not a real condition of real world scenarios, where only \emph{partial knowledge} can be ensured.
For this reason, our tool is able to compute a plan even when only partial information of the environment is available, as seen in \cite{roy2006planning,du2012robot,diaz2001exploring}.
However, the novelty of our work consists in fuse all this features, exploiting a \emph{decentralized} methodology.
This kind of approaches are not yet studied in detail, due to there are only a few planners managing this issues \cite{guo2015multi}.

\textbf{Organization.} 
Section~\ref{sec:limitations} introduces robotic applications by highlighting the status of current planners.
Section~\ref{sec:approach} describes the \toolName\ approach.
Section~\ref{sec:tool} presents the \toolName\ tool.
Section~\ref{sec:conclusion} concludes with final remarks.

\section{Research approach}
\label{sec:approach}
In this project we are working from the academic point of view research-wise but we are also working closely with the industry.
As stated before, the work developed for this PhD project is embedded in the Co4Robots European project.
The main goal of this project is to deploy a robotic application in a ``domestic" environment such as hospitals, hotels, airports, etc.
These robotic applications must be able to accomplish complex missions with a systematic, real-time, decentralized methodology.
Furthermore, a robotic application could be composed by a team of potentially heterogeneous robots.
The aforementioned environments will be considered as dynamic and will also count with the presence of human beings.
For this reason, the robots must have integrated a set of perceptual capabilities that enables them to localize themselves and estimate the state of their highly dynamic environment in the presence of strong interactions and in a collaborative manner.
Robots must not only interact between them, but also with human beings.

In order to validate the code and artifacts developed for Co4Robots, the committee defined a set of study cases for the project proposal.
Our framework builds upon various cases of base interactions between agent pairs of different types. 
The considered inter-agent interactions are:
\begin{itemize}
\item[Case A] physical guidance by a human for the transportation of an object carried by a robot;
\item[Case B] collaborative grasping and manipulation of an object by two agents;
\item[Case C]collaborating mobile platform and stationary manipulator to facilitate loading and unloading tasks onto the mobile platform; and
\item[Case D] information exchange between a human giving orders and a robotic agent. 
\end{itemize}

On the other hand, it is important to decouple the research made just for the project and the one intended for the whole PhD.
While the single-robot part can be more driven by the Co4Robots goal, the second part is expected to reach far beyond results than the intended for the project.
For example, the multi-robot choreography for Co4Robots is limited by its study cases to two robots.
However, for our own research we plan to manage a team of them acting under some defined collaborative strategies and adapting to emergent properties.



\section{Engineering robotic applications}
\label{sec:single}
After the research of the current state of the art we proceeded with RQ2, that will be addressed during the first part of the PhD as well.
In order to answer it while explaining our proposed process we divided the development of our framework into different tasks, that correspond with the items stated in Section~\ref{sec:introduction}.

\subsection{Software architecture}
\label{sec:softwarearch}

We propose our software architecture, Self-adaptive dEcentralised Robotic Architecture (SERA) for the framework.
As its name indicates, it supports a real-time decentralized robot coordination to accomplish missions with teams of robots. 
Furthermore, it is self-adaptive, responding to external and internal events by computing new strategies to achieve the desired goals.
SERA is inspired by and extends concepts of existing proposals for robot software architectures from the literature. 
Specifically, we inspected architectures identified by a mapping study of Ahmad et al.~\cite{Ahmad201616}, which investigated software architectures for robotics systems to identify and analyze the relevant literature based on 56 peer-reviewed papers.
The aforementioned architecture is three layers architecture that is strongly influenced by the well-known work of Kramer and Magee~\cite{kramer}.
Furthermore, it is a component-based type of architecture, so functionalities of the system are encapsulated in modules called ``components".
It is important to remark that the aim of our project is to build a system that can be easily used by not technical users, so we had to define a way for them to command the missions to the robotic team.
For this reason we added a central station that is just used during design-time in order to allocate a graphical interface to be used by a final user.

We defined SERA by first conceiving an architecture for a single robot. 
Then, we extended and refined it in order to iteratively extend and refine the architecture towards enabling communication among robots and collective adaptations. 
Thus, all the robots have a instantiation of the reference architecture but are also able to communicate and share information with the rest of the team, making possible the collective adaptation. 

SERA was already tested during an Integration Meeting of the project, where it demonstrate that can support the performance of a robot achieving different complex missions ---i.e. collaborative transportation with an human being, autonomous driving in a dynamic environment.

%\begin{figure*}[!t]
%\begin{center}
%\includegraphics[width=1\linewidth]{Figures/InstanceMultiRobot_Graffle.pdf}
%\caption{Software architecture.}
%\label{fig:arch}
%\end{center}
%\end{figure*}

\subsection{Software platform}

As explained before, the Software platform will integrate the Software architecture, all the tools and software created by developers and the Configuration facilities.
The platform is also a collection of the components that compose the architecture, a kind of library.
In our project the components of our architecture represent ROS~\cite{Quigley2009} nodes and packages, so the platform itself should be based on this middleware.
All this components are developed abstracting the communication problems since we rely on the interfaces defined in the architecture.
It not only significantly reduces the complexity of the code but also triggers the modularity of our system making possible exchanging the components based on the context.

We not only plan to control the performance of the system that is running in each robot but also its behaviour.
Thus, the usage of a high-level behavior engine, flexibly applicable to numerous systems and scenarios is mandatory.
FlexBe~\cite{Schillinger2016} not only provides a way of defining the behaviour of the robot in different scenarios (as of the study cases) by can also be used for defining the work flow.
It also provides a graphical interface that simplifies enormously these tasks.
FlexBe encapsulates functionalities of the robotic application, as our architecture does within components, and provides a way of orchestrate them so we keep the modularization of our system.
As with ROS, an instantiation of FlexBe will be deployed in every robot.
The integration of FlexBe is driven for the necessity of a software development methodology as MDE in our project.
It improves the modularity, variability and reusability of our system facilitating the development of the software for animating robotic systems through the creation of reusable robot building blocks with well-defined interfaces and properties.

Finally, in order to communicate each robot with its teammates we implemented an approach based on ROS+REST.
So, using a suitable component that works as an interface we are able to send messages in form of services between robots.
%In this way, each robot has an instance of ROS running in their own local environment so we can deploy a whole team of robots avoiding a central master node and the problems related with this approach (i.e. bottleneck issues, less robustness facing failures of a node, etc.), specially working with the ROS middleware.

\subsection{Configuration facilities}

Since the components of our architecture are exchangeable our next short-term goal is to define configuration facilities that can be applied to our system depicted in the architecture.
It will allow to our applications to support two things:

\begin{enumerate}
\item Being customizable at design-time, so we can configure its components based on the requirements of our context (i.e. hardware installed in each robot, environment where they will be deployed, etc.)
\item To self-adapt or self-configure at run time, so each robot can apply changes in its configuration based on emergent events of the environment or failures of their system.
\end{enumerate}

In order to do so we will implement pluginlib~\footnote{http://wiki.ros.org/pluginlib}, a package that uses the ROS build infrastructure and provides tools for writing and dynamically load plugins.





\section{Supporting the choreography of robotic applications}
\label{sec:multi}
The future work will be focused on trying to find an answer to RQ3 during the second part of the research.
In order to achieve it, a detailed study of the current state of multi-robot choreography regarding emergent properties and collaborative adaptation must be performed.
%Nevertheless, since our plan is to continuously integrate all the obtained knowledge into the project, the framework developed during the first division of the project will be used and tested through this part as well.
%Furthermore, due to the tools created for the robot orchestration will be tested on real robots we still will need to deploy the software platform and its functionalities on each of them.
An architectural overview of the whole research is depicted in Figure~\ref{fig:overview}.
In this figure the expected final system is represented in a schematic way.
The aforementioned conceptual and temporal division of the research is expressed with dot-lined boxes.
The contents of the box concerning RQ1 comprehend everything since it represents a broad study.
%Then, the research concerning RQ2 is more focused.
The box that represents RQ2 contains a conceptual representation of a robotic team represented by a group of folded boxes that in turn contain the framework intended for each robot.
The communication methodology is also represented.
Then, the techniques that we plan to study in order to perform the choreography are contained in the RQ3 box.
%Note that the contents regarding RQ2 are embedded into RQ3.

During the second part of this PhD project, techniques for supporting the correct choreography of multiple robots will be studied and applied.
Robots should collaborate in a team to accomplish complex missions because often adaptations performed by a single robot are not sufficient to accomplish a specified mission. 
Tasks may need to be reassigned to other robots, or a team of robots needs to be reconfigured. 
Collaboration and interaction of robots among themselves and with the environment could lead to emergent properties representing unexpected behaviors. Emergent properties can be beneficial, neutral, or even harmful ---for instance, when they hamper safety or the mission accomplishment.

\begin{figure}[!t]
\begin{center}
\includegraphics[width=1\linewidth]{Figures/research.pdf}
\caption{Architectural overview of the final system.}
\label{fig:overview}
\end{center}
\end{figure}

\section{Related work}
\label{sec:related}
\emph{Decentralized solutions.}
%Kind of state of the art, relate de tool 

\sergio{Not sure if we should add some planner tools as V-Rep http://www.coppeliarobotics.com/, ROS Moveit http://moveit.ros.org/, Gazebo http://gazebosim.org/ or look for papers presenting tools}

Decentralized planning problem has been studied for known environments~\cite{schillinger2016decomposition,guo2015multi,tumova2016multi}.
However, planners for partially known environments do not usually employ decentralized solutions~\cite{roy2006planning,du2012robot,diaz2001exploring}. 

\emph{Dealing with partial knowledge in planning.}
Planning in partially known environments is handled in different ways. 
\begin{enumerate*}
\item Several works (e.g.,~\cite{ding2011ltl,kurniawati2011motion,wolff2012robust,du2012robot,Roy2006,chen2012ltl,nikou2017probabilistic,7078886,7139350,narayanan2015task}) consider probabilities within the planning algorithm.
Most of these works  treat partial information by modeling the robotic application using some form of \emph{Markov decision processes} (MDP).
In some of these works~\cite{ding2011ltl,chen2012ltl} transitions of the robots are associated with probabilities which indicate the probability of reaching the destination of the transition given that an action is performed.
In other works~\cite{wolff2012robust}, transition probabilities are not exactly known but are known to belong to a given uncertainty sets.
Finally, several works~\cite{kurniawati2011motion,Roy2006} consider partially observable Markov decision processes.
All these approaches generally generate plans that maximize the worst-case probability of satisfying a mission.
Differently, our work does not consider probabilities.
\item Several works (e.g,~\cite{lahijanian2016iterative,livingston2012backtracking,l2014safety,nie2016searching,7139412}) studied how to change the planned trajectories when unknown obstacles are detected or when obstacles move in a unpredictable way.
In this case, the used underlying model is some sort of \emph{hybrid model}, i.e., models in which finite state machines are combined with differential equations. 
In~\cite{lahijanian2016iterative}, to plan trajectories the authors use a high-level planner that exploits an abstraction of the hybrid system and the mission to compute high-level plans. 
The low-level planner uses the dynamics of the hybrid system and the suggested high-level plans to explore the state space for feasible solutions.
Every time an unknown obstacle is encountered, the high-level planner modifies the coarse high-level plan online by accounting for the geometry of the discovered obstacle. 
Within this framework, \toolName\ can be considered as a high-level planner that is able to use an abstraction of the hybrid system that contains partial information, i.e., encode unknown obstacles.
\item Some  approaches  analyzed how to update plans when new information about known model of a robotic application is detected (e.g.,~\cite{guo2015multi}). 
Differently, in our approach portions of the model of the robotic application are partially known,  partial knowledge is reduced as true and false evidence about partial information is detected.
Other works (e.g.,~\cite{7139310}), aim at detecting how to explore totally unknown environments.
\item 
Plan synthesis is a particular instance of controller synthesis. 
Controller  synthesis (e.g.,~\cite{cassandras2009introduction,D'ippolito:2013:SNE:2430536.2430543}) aims at finding a component, usually indicated as controller or supervisor, that ensures property satisfaction for all the possible system executions.
Differently, plan synthesis aims at finding a single execution, i.e., a plan that ensures property satisfaction.
The controller synthesis  is usually (\cite{kress2009temporal,wongpiromsarn2009receding,chen2012ltl,livingston2012backtracking,guo2013revising}) performed by solving a two player game between robots and their environment.
The goal is to find a strategy the robots can use that allows always winning the game.
Differently, in our case the planning algorithm ensures that there is a way of completing the \emph{single} (possible) plan that satisfies the property of interest. 
\item \toolName\ can be classified on the boundary between reactive synthesis~\cite{chen2012ltl,livingston2012backtracking,thomas2002automata} techniques and iterative planning~\cite{guo2013revising,maly2013iterative}. 
As reactive synthesis techniques, \toolName\ constructs a control strategy that accounts for every possible variation in the environment, but the computed plan does not allow always winning the  \emph{two player game} between the robots and their environment.
As  iterative planning, a new plan is computed on-the-fly when new information is available.
\end{enumerate*}


\section{Validation}
\label{sec:validation}
%In order to validate the code and artifacts developed for Co4Robots, the consortium defined a set of study cases in the project proposal.
%Our framework builds upon various cases of base interactions between agent pairs of different types. 
%The considered inter-agent interactions are:
%\begin{itemize}
%\item[Case A] physical guidance by a human for the transportation of an object carried by a robot;
%\item[Case B] collaborative grasping and manipulation of an object by two agents;
%\item[Case C]collaborating mobile platform and stationary manipulator to facilitate loading and unloading tasks onto the mobile platform; and
%\item[Case D] information exchange between a human giving orders and a robotic agent. 
%\end{itemize}

In order to validate the code and artifacts developed for Co4Robots we defined three different approaches than can be followed depending on the artifact that we want to obtain.
The first approach consists in getting feedback from stakeholders and practitioners; this has been done for instance through presentations of our work during meetings and consortiums.
For the second, we use simulation tools that allow us to validate our artifacts before implementing them into real robots.
Finally, the third one consists in validation with real robots in real-world scenarios.
For example, the outline of the approach that we followed for validating SERA is:
\begin{enumerate}
\item Personal work alternated with internal meetings.
\item First validation by means of robot simulators\footnote{http://gazebosim.org/}.
\item Presentation of achieved work to the Co4Robots committee and collection of feedback.
\item Correction of work based on feedback.
\item Validation during Milestones and Integration Meetings with real robots in real-world scenarios.
\item Documentation for deliverables requested for every task within each Workpackage of Co4Robots.
\end{enumerate}


%The case studies are the best tool for testing and validating most of the results obtained in this research.
%For each case study, the Co4Robots consortium holds a \emph{Milestone meeting} in order to test all the developed tools.
%It allow us not only to test our research outcomes in real robots working in real-world scenarios but also to discuss with the rest of the developing stakeholders involved in the project and to collect valuable feedback.

%The outline of the approach that we are currently follow for validation of our research is as follows:

%In this case we combined the three stated validation approaches.
%However, there are other cases ---such as the validation of the collaborative adaptation algorithms--- where we plan to validate our artifacts starting from the second approach and then following with the third one since they do not need validation from the Co4Robots stakeholders.







\section{Conclusions}
\label{sec:conclusion}
Software engineering can be the key technology needed for the improvement of applications developed for robotic systems.
Robots are nowadays a trend, but without well-defined engineering process for the researchers to follow most of the projects are started from scratch and not time neither cost efficient.
In this project we aim to tackle those problems while validating the premises and results with real-world scenarios where real robots work in a collaborative way.

Furthermore, we plan to perform a detailed research in the state of the art

\section*{Acknowledgments}
\label{sec:Acknowledgments}

EU H2020 Research and Innovation Programme, grant 731869 (Co4Robots).


\balance

\bibliographystyle{ACM-Reference-Format}
\bibliography{sigproc} 

\end{document}
