\documentclass[sigconf,review, anonymous]{acmart}
\acmConference[ICSE 2018]{40th International Conference on Software Engineering}{May 27--June 3, 2018}{Gothenburg, Sweden}
\acmYear{2018}

\usepackage{booktabs} % For formal tables
\usepackage{centernot}
\usepackage{algorithm}
\usepackage{algorithmic}
\usepackage{amsmath,amssymb,amsfonts}
\usepackage{balance}

\newtheorem{remark}{Remark}
\newtheorem{problem}{Problem}
\newtheorem*{running*}{Running example}
\usepackage[inline]{enumitem}
\settopmatter{printfolios=true}


\usepackage[xcolor=orange]{changes}
\definechangesauthor[name={Claudio Menghi},color=orange]{CM}
\definechangesauthor[name={Sergio Garcia},color=red]{SG}
%\definechangesauthor[name={Patrizio},color=blue]{PP}
\usepackage[colorinlistoftodos,prependcaption,textsize=tiny]{todonotes}
%\newcommand{\sergio}[1]{\todo[color=blue]{\textsf{SG} #1}}

\newboolean{showcomments}
\setboolean{showcomments}{true} % toggle to show or hide comments
\ifthenelse{\boolean{showcomments}}
{\newcommand{\nb}[2]{
  \fcolorbox{black}{yellow}{\bfseries\sffamily\scriptsize#1}
  {\sf\small$\blacktriangleright$\textit{#2}$\blacktriangleleft$}
 }
 \newcommand{\version}{\emph{\scriptsize$-$working$-$}}
}
{\newcommand{\nb}[2]{}
 \newcommand{\version}{}
}
\newcommand\patrizio[1]{\nb{Patrizio}{#1}}
\newcommand\claudio[1]{\nb{Claudio}{#1}}
\newcommand\sergio[1]{\nb{Sergio}{#1}}
\newcommand\jana[1]{\nb{Jana}{#1}}

\newcommand{\Sync}{\ensuremath{Meet}}
\newcommand{\cla}[1]{\textcolor{red}{{#1}}}
%%%shortcuts
\newcommand{\robotindex}{n}
\newcommand{\robotindexp}{m}
\newcommand{\setrobot}{\ensuremath{\mathcal{R}}}
\newcommand{\T}{\ensuremath{r}} %transition system

\newcommand{\powerset}{\raisebox{.15\baselineskip}{\Large\ensuremath{\wp}}}
\newcommand{\robot}{\ensuremath{\T}}

\newcommand{\INPUT}{\textbf{Input} }
\newcommand{\OUTPUT}{\textbf{Output} }

\newcommand{\AP}{\Pi} %atomic propositions
\newcommand{\service}{\pi}
\newcommand{\behavior}{\ensuremath{\mathcal{B}}} %Buchi automaton
\newcommand{\trace}{\ensuremath{\mathcal{T}}}
\newcommand{\A}{\ensuremath{\mathcal{A}}} %automaton
\newcommand{\B}{\ensuremath{\mathcal{B}}} %Buchi automaton
\newcommand{\BA}{B\"uchi automaton }
\renewcommand{\P}{\mathcal{P}} %product automaton
\newcommand{\R}{\mathcal{R}} %rabin automaton
\newcommand{\init}{\ensuremath{init}}
\newcommand{\pref}{\mathit{pref}}
\newcommand{\currs}{\mathfrak{s}}
\newcommand{\currq}{\mathfrak{q}}
\newcommand{\dur}{\Delta}
\newcommand{\wait}{\mathit{wait}}
\newcommand{\sync}{\mathit{sync}}
\newcommand{\nosync}{\mathit{nosync}}
\newcommand{\TS}{\ensuremath{\T=(S,} \ensuremath{\init}, \ensuremath{{A},} \ensuremath{\AP,} \ensuremath{T,} \ensuremath{\Sync,} \ensuremath{L)}}
\newcommand{\PTS}{\ensuremath{\T=(S,} \ensuremath{\init},   \ensuremath{{A},} \ensuremath{\AP,}    \ensuremath{T,}   \ensuremath{T_p,}  \ensuremath{\Sync}, \ensuremath{\Sync_p}, \ensuremath{L} )}
\newcommand{\PTSone}{\ensuremath{\T'=(S',} \ensuremath{\init',} \ensuremath{{A'},} \ensuremath{\AP',} \ensuremath{T',}  \ensuremath{T'_p,} \ensuremath{\Sync' )}}

\newcommand{\ra}{$\rightarrow$}
\newcommand{\ugh}[1]{\textcolor{red}{\uwave{#1}}} % please rephrase
\newcommand{\ins}[1]{\textcolor{blue}{\uline{#1}}} % please insert
\newcommand{\del}[1]{\textcolor{red}{\sout{#1}}} % please delete
\newcommand{\chg}[2]{\textcolor{red}{\sout{#1}}{\ra}\textcolor{blue}{\uline{#2}}}

\newcommand{\TSprime}{{\T'=(S',\init',{A}',T', \Sync')}}

\newcommand{\TSoneprime}{{\T_1=(S'_1, \init'_{1},{A}'_1,T'_1, \Sync')}}

\newcommand{\TSone}{{\T_1=(S_1,\init_{1},{A}_1,T_1, \Sync_1)}}
\newcommand{\TStwo}{{\T_1=(S_2, \init_{2},{A}_2,T_2, \Sync_2)}}
\newcommand{\TSproduct}{{(S_1 \times S_2,\langle \init_{1}, s_{2,\init} \rangle,{A}_2 \cup {A}_2,T, \Sync)}}
\newcommand{\TStwoprime}{{\T_1=(S'_2,\init'_{2},{A}'_2,T'_2)}}
\newcommand{\robotpar}{\ensuremath{\T_\robotindex=} \ensuremath{(S_\robotindex,} \ensuremath{\init_{\robotindex},} \ensuremath{{A}_\robotindex,} \ensuremath{\AP_\robotindex ,} \ensuremath{T_\robotindex,} \ensuremath{\Sync_\robotindex)}}
\newcommand{\robotp}[1]{\ensuremath{\T_{#1}=} \ensuremath{(S_{#1},} \ensuremath{\init_{#1},} \ensuremath{{A}_{#1},} \ensuremath{\AP_{#1} ,} \ensuremath{T_{#1},} \ensuremath{\Sync_{#1})}}
\newcommand{\partialrobotp}[1]{\ensuremath{\T_{#1}}=(\ensuremath{S_{#1}}, \ensuremath{\init_{#1},} \ensuremath{{A}_{#1},} \ensuremath{\AP_{#1},} \ensuremath{T_{#1},} \ensuremath{T_{p,#1},} \ensuremath{\Sync_{#1},} 
\ensuremath{\Sync_{p,#1}}, \ensuremath{{L}_{#1}})}


\newcommand{\PTSpar}{\ensuremath{\T_\robotindex=(S_\robotindex,} \ensuremath{\init_{\robotindex},} \ensuremath{{A}_\robotindex,} \ensuremath{\AP_\robotindex ,} \ensuremath{T_\robotindex,} \ensuremath{T_{p,\robotindex},} \ensuremath{\Sync_\robotindex}, 
\ensuremath{\Sync_{p,\robotindex}, \ensuremath{L}_\robotindex)}}

\newcommand{\network}{\ensuremath{\N}}
\newcommand{\robotapplication}{\ensuremath{\N}}
\newcommand{\robotset}{\network = \{\ensuremath{\robot_1,} \ensuremath{\robot_2, \ldots}, \ensuremath{\robot_N\}}}
\newcommand{\robotsetdef}{\{\ensuremath{\robot_1,} \ensuremath{\robot_2, \ldots}, \ensuremath{\robot_N\}}}
\newcommand{\robotsetdefp}{\{\ensuremath{\robot'_1,} \ensuremath{\robot'_2, \ldots}, \ensuremath{\robot'_N\}}}
\newcommand{\property}{\ensuremath{\phi}}
\newcommand{\missions}{\ensuremath{\Phi}}
\newcommand{\propertiesset}{\ensuremath{\Phi =} \{\ensuremath{\property_1,} \ensuremath{\property_2, \ldots}, \ensuremath{\property_N\}}}


\newcommand{\TSi}{{\T_i=(S_i,\init_{i},{A_i} ,T_i)}}
\newcommand{\N}{\mathcal{H}}
\newcommand{\M}{{M}}
\newcommand{\model}{\mathcal{M}}
\newcommand{\I}{\mathcal{I}}
\newcommand{\D}{{D}}
\renewcommand{\O}{\mathcal{O}}

\newcommand{\toolName}{MAPmAKER}
\newcommand{\APs}{\mathbf{\Pi}}
\newcommand{\Lang}{\mathcal{L}} %language
\newcommand{\Set}{\mathsf{S}} %set
\newcommand{\Spec}{\mathbf{\Phi}}
\newcommand{\Epsilon}{\mathcal{E}}
\renewcommand{\i}{\iota}
\newcommand{\Nat}{\mathbb{N}} %natural numbers
\newcommand{\Real}{\mathbb{R}}
\newcommand{\Next}{\mathsf{X}}
\newcommand{\Until}{\mathsf{U}}
\newcommand{\Always}{\mathsf{G}}
\newcommand{\Event}{\mathsf{F}}
\newcommand{\false}{\mathit{false}}
\newcommand{\true}{\mathit{true}}
\newcommand{\trueval}{\ensuremath{\top}}
\newcommand{\falseval}{\ensuremath{\bot}}
\newcommand{\maybe}{\ensuremath{?}}
\renewcommand{\epsilon}{\varepsilon}
\newcommand{\prop}{\pi}
\newcommand{\ie}{{i.e., }}
\newcommand{\eg}{{e.g., }}
\newcommand{\progressive}{\varpi}
\newcommand{\move}{\mathit{move}}
\newcommand{\h}{h}
\renewcommand{\H}{H}
\newcommand{\parti}{\mathit{P}}
\newcommand{\Alpha}{\mathbf{\Sigma}}
\renewcommand{\mod}{\mathrm{\, mod \, }}
\newcommand{\suc}{\mathit{succ}}
\newcommand{\dist}{\mathrm{dist}}
\newcommand{\proj}{\mathrm{proj}}
\newcommand{\parspace}{\vskip 0.05in}


\makeatletter
\newcommand*{\da@rightarrow}{\mathchar"0\hexnumber@\symAMSa 4B }
\newcommand*{\da@leftarrow}{\mathchar"0\hexnumber@\symAMSa 4C }
\newcommand*{\xdashrightarrow}[2][]{%
  \mathrel{%
    \mathpalette{\da@xarrow{#1}{#2}{}\da@rightarrow{\,}{}}{}%
  }%
}
\newcommand{\xdashleftarrow}[2][]{%
  \mathrel{%
    \mathpalette{\da@xarrow{#1}{#2}\da@leftarrow{}{}{\,}}{}%
  }%
}
\newcommand*{\da@xarrow}[7]{%
  % #1: below
  % #2: above
  % #3: arrow left
  % #4: arrow right
  % #5: space left 
  % #6: space right
  % #7: math style 
  \sbox0{$\ifx#7\scriptstyle\scriptscriptstyle\else\scriptstyle\fi#5#1#6\m@th$}%
  \sbox2{$\ifx#7\scriptstyle\scriptscriptstyle\else\scriptstyle\fi#5#2#6\m@th$}%
  \sbox4{$#7\dabar@\m@th$}%
  \dimen@=\wd0 %
  \ifdim\wd2 >\dimen@
    \dimen@=\wd2 %   
  \fi
  \count@=2 %
  \def\da@bars{\dabar@\dabar@}%
  \@whiledim\count@\wd4<\dimen@\do{%
    \advance\count@\@ne
    \expandafter\def\expandafter\da@bars\expandafter{%
    }%
  }%  
  \mathrel{#3}%
  \mathrel{%   
    \mathop{\da@bars}\limits
    \ifx\\#1\\%
    \else
      _{\copy0}%
    \fi
    \ifx\\#2\\%
    \else
      ^{\copy2}%
    \fi
  }%   
  \mathrel{#4}%
}

\everymath{\vadjust{\nobreak\null}}

\makeatletter
\def\old@comma{,}
\catcode`\,=13
\def,{%
  \ifmmode%
    \old@comma\discretionary{}{}{}%
  \else%
    \old@comma%
  \fi%
}
\makeatother

\def\HiLi{\leavevmode\rlap{\hbox to \hsize{\color{yellow!50}\leaders\hrule height .8\baselineskip depth .5ex\hfill}}}

\newif\ifextended
\extendedfalse
%\extendedtrue

\ifextended
\newcommand{\extended}[1]{\textcolor{red}{#1}} 
\newcommand{\notextended}[1]{}
\else
\newcommand{\extended}[1]{}
\newcommand{\notextended}[1]{#1}
\fi

\begin{document}
\title{MAPmAKER: A Tool for Performing Multi-Robot LTL Planning Under Uncertainty}


%\author{Sergio Garc\'{i}a}
%\affiliation{%
%  \institution{University of Gothenburg}
%  \streetaddress{P.O. Box 1212}
%  \city{Gothenburg} 
%  \state{Sweden} 
%  \postcode{405 30}
%}



\begin{abstract}
Robot applications are increasingly asking for decentralized techniques that allow for tractable automated planning. 
Furthermore, those applications can be deployed in dynamic environments where its uncertainty must be handled.
Typically, environments where human beings are involved can just provide a partial knowledge of its model, i.e. the current state of a door between two rooms in uncertain. 

Our proposed tool, \toolName tackle the limitations that current planning techniques are used for teams of robots: 
\begin{enumerate*}
\item it decomposes the robotic team into subclasses, avoiding the not scalable centralized approach;
\item it considers complex-high level missions given in temporal logic;
\item it is able to work also with only partial knowledge of the environment, performing possible plans.
\end{enumerate*}

\end{abstract}

%
% The code below should be generated by the tool at
% http://dl.acm.org/ccs.cfm
% Please copy and paste the code instead of the example below. 
%
%\begin{CCSXML}
%<ccs2012>
% <concept>
%  <concept_id>10010520.10010553.10010562</concept_id>
%  <concept_desc>Computer systems organization~Embedded systems</concept_desc>
%  <concept_significance>500</concept_significance>
% </concept>
% <concept>
%  <concept_id>10010520.10010575.10010755</concept_id>
%  <concept_desc>Computer systems organization~Redundancy</concept_desc>
%  <concept_significance>300</concept_significance>
% </concept>
% <concept>
%  <concept_id>10010520.10010553.10010554</concept_id>
%  <concept_desc>Computer systems organization~Robotics</concept_desc>
%  <concept_significance>100</concept_significance>
% </concept>
% <concept>
%  <concept_id>10003033.10003083.10003095</concept_id>
%  <concept_desc>Networks~Network reliability</concept_desc>
%  <concept_significance>100</concept_significance>
% </concept>
%</ccs2012>  
%\end{CCSXML}
%
%\ccsdesc[500]{Computer systems organization~Embedded systems}
%\ccsdesc[300]{Computer systems organization~Redundancy}
%\ccsdesc{Computer systems organization~Robotics}
%\ccsdesc[100]{Networks~Network reliability}


%\keywords{ACM proceedings, \LaTeX, text tagging}


\maketitle

\section{Introduction}
\label{sec:introduction}
Service robots are increasingly being involved in human lives. 
They are increasingly used in environments such as houses, airports, hospitals, and offices for performing navigation, transportation, and manipulation tasks. 
The World Robotic Survey~\cite{wrs:online} estimated 35 million indoor service robots to be sold by 2018, accumulating a sales value of \$12 billion since 2015. 
%The global sales of household and personal robots is expected to grow by 23.5\% per year~\cite{sheng:online}. 
This increase is accompanied with huge progress in robot technology. %, especially in image processing, planning, control, and collaboration. 
Software engineering is key to sustaining this new technology.

A robot typically performs specialized tasks; however, some tasks are highly complex and require a team of robots, whose capabilities (e.g., perception, manipulation, and actuation) are coordinated and supervised. 
Such teams also need to adapt to changes, such as of the environment, of the desired tasks, or of the robot (e.g., hardware failures). 
These demands drive the complexity of robot control software relying on appropriate software architectures. 
To tackle this complexity, we need to rethink design processes~\cite{Lee2008} by properly managing system integration and raising the abstraction levels, addressing qualities like evolvability~\cite{Perez2008}, configurability~\cite{Gamez2013563}, scalability and dependability.

%The PhD project presented in this paper 
Our work is involved in the Co4Robots~\footnote{http://www.co4robots.eu/} European project.
Being part of this project allow us not only to formulate but also to validate our research questions in real-world scenarios.
%According to~\cite{roadmap}: ``Usually there are no system development processes (highlighted by a lack of overall architectural models and methods). 
%This results in the need for craftsmanship in building robotic systems instead of following established engineering processes".
According to~\cite{roadmap} there are no system development processes for robotic applications.
%In fact, most of the robotic applications developed nowadays have to be started from scratch without following well-engineered methods.% to help in the process.
In fact, robotic applications nowadays are started from scratch without following well-engineered methods.
In this context the aim of the Co4Robots project is to establish systematic engineering process to facilitate the development of the software for animating robotic systems through the creation of reusable robot building blocks with well-defined interfaces and properties. 

The research conducted during the present project will be split in two parts.
The first one defines the best practices for engineering software robotic applications.
During this period we studied the current software engineering practices for both single and multi-robotic systems.
Furthermore, throughout this part of the research we will develop the platform that will be allocated within every robot of our applications.
An instance of our \emph{Software platform} will be deployed on each robot.
It will integrate the \emph{Software architecture} of the whole system and  the \emph{Configuration facilities}, which provide the required tools for configuring our architecture.% both at design and run-time.

The second part of the research aims to support the \emph{choreography} of robotic applications.
It is considered future work that will be tackled once the platform is completely addressed.
In this context, choreography means the way of representing and controlling the interactions between multiple services of a system in a decentralized way.
%The main expected outcome of this part is to perform 
Our goal is to perform the choreography of a deployed team of potentially heterogeneous robots in dynamic environments with the presence of human beings.
In order to do so, issues as \emph{Emergent properties} ~\cite{DeAngelis2016} and selecting the most suitable \emph{Collaborative adaptation}~\cite{Yan2013} techniques must be addressed.

\textbf{Research Questions.} 
Based on the division of the project, we state the following research questions:
\begin{itemize}
\item[RQ1] Which are the current software engineering practices for engineering robotic applications and which are their limitations?
\item[RQ2] Which software engineering practices can be developed to improve the process of engineering robotic applications?
\item[RQ3] Which are the applicable strategies to manage a heterogeneous robotic application with only partial knowledge of a dynamic environment?
\end{itemize}

\textbf{Contributions.} 
Our contributions are listed in the following:

\begin{enumerate}
\item Definition of a software architecture able to structure a robotic team;
%\item Validation of the architecture in a real-world scenario
\item Implementation of a software platform where all the algorithms and tools developed can be plugged in;
\item Definition of configuration mechanisms to enable \emph{start-up configuration} and \emph{run-time configuration};
\item Integration of an approach based on ROS+REST for the internal communication between robots;
\item Development of the algorithms in charge of managing the robotic team, based on management of emergent properties and selection of collaborative adaptation techniques.
%\end{enumerate}
\end{enumerate}

At this moment the software architecture and the communication mechanisms are already developed.
The software platform is being continuously developed and we intend to start dealing with the configuration facilities in the near future.
The development of the algorithms in charge of managing the robotic team are planned to be addressed during the second part of the project.

\textbf{Organization.} 
Section~\ref{sec:approach} describes our research approach and answers the first RQ. %of the current work and answers the first RQ.
In Section ~\ref{sec:single} we present the process that we follow for engineering robotic applications and answer RQ2.
Then, in Section~\ref{sec:multi} we explain our plan for managing a robotic team while answering RQ3.
In Section~\ref{sec:related}, we introduce different works with a similar scope and position our research.
Section~\ref{sec:validation} explains our validation plan.
Section~\ref{sec:conclusion} concludes with final remarks.

%\section{Related work}
%\label{sec:related}
%In this section we discuss related work with respect with our two proposed Research Questions.

Our robotic application will be based on ROS~\cite{Quigley2009} and some functionalities will be build on top of it. 
ROS is an open-source meta-operating system for robots. 
It provides a communication layer above the Linux host operating system that supports the execution of components in a distributed system. 
ROS offers message-based peer-to-peer communication infrastructure supporting the integration of independently developed software components, called ROS nodes, that are organized into a graph.

The benefits for using ROS are many and one of them is the flexibility that this tool provides to the developers.
However, this flexibility could result in a development process based on ad-hoc solutions rather than being based on a systematic engineered approach. 
Obviously, it decreases the modularity and reusability of the developed system and makes its development process to take longer due to many of its applications will have to be generated from scratch. 
Furthermore, ROS has some limitations, some of them recognized by their developers~\footnote{http://design.ros2.org/articles/why\_ros2.html}.
ROS2 is supposed to solve these previous problems and to substitute ROS1 in a near future, but since it was just released and there are not yet all the contents that were available for ROS1 we opted for keep using ROS1.

Architectures
	Decentralized
	Multi-agent
	Not SOA, microservices, cloud container

Collaborative adaptation

%
%\section{Research approach}
%\label{sec:approach}
%The plan within this project is to split the work not only regarding topics but time in two parts:
\begin{enumerate*}
\item the first part, focused on the study of current practices for engineering robotic applications (RQ1) and the development of the framework allocated within each robot (RQ2); and
\item the second part, focused on collaboratively managing a team of robots (RQ3).
\end{enumerate*}

This project is being developed closely with the industry because it is embedded in the Co4Robots European project.
The main goal of this project is to deploy a robotic application in a ``domestic" environment such as hospitals, hotels, airports, etc.
These environments will be considered as dynamic and will also count with the presence of human beings.
We consider that robotic applications must be able to accomplish complex missions with a systematic, real-time, decentralized methodology.
Furthermore, a robotic application could be composed by a team of potentially heterogeneous robots.
For this reason, the robots must have integrated a set of perceptual capabilities that enables them to localize themselves and estimate the state of their highly dynamic environment in the presence of strong interactions and in a collaborative manner.
That is, robots must not only interact between them, but also with human beings.

In order to learn which are the current practices for developing robotic applications we performed an extensive research in the field.
We also plan to conduct empirical studies such as interviews with different companies, starting with the industrial partners of Co4Robots.
With this steps we expect to answer RQ1 and also to figure out which are the limitations of such practices.  

A not negligible task within this project is to decouple the research made just for Co4Robots and the one intended for the whole PhD.
While the framework development outcomes is common for both our own research and the Co4Robots' , our outcomes of the collaborative adaptation part are expected to reach far beyond results than the intended by Co4Robots.
%For example, for our own research we plan to manage a team of robots acting under some defined collaborative strategies and adapting to emergent properties.




%
%\section{Current status}
%\label{sec:current}
%Currently, our main focus lays on the RQ1, learning, defining and differencing between software engineering practices regarding single-robot and multi-robot systems.
Our software architecture, Self-adaptive dEcentralised Robotic Architecture (SERA) is already defined.
As its name indicates, it supports a real-time decentralized robot coordination to accomplish missions with teams of robots. 
Furthermore, it is self-adaptive, responding to different changes by computing new strategies to achieve the desired goals.
SERA was already tested during an Integration Meeting of the project, where it demonstrate that can support the performance of a robot achieving different complex missions ---i.e. collaborative transportation with an human being, autonomous driving in a dynamic environment.

The aforementioned architecture follows the component-based style, so the main robotic functionalities are encapsulated in different modules or "components".
All this components are developed abstracting the communication capabilities since we rely on the interfaces defined in the architecture.
It not only significantly reduces the complexity of the code but also triggers the modularity of our system making possible exchanging the components that conform our architecture.

Since the components of our architecture are exchangeable our next step is to define configuration facilities in order to allow the customization of the robotic application during design-time or its self-adaptation at run-time.
\sergio{cita a rosplug lib?}

Finally, the communication approach based on ROS+REST is already implemented, allowing the famous robotic middleware to share information between different robots in a decentralized way using services.


%
%\section{Future work and directions}
%\label{sec:future}
%As stated before, we plan to split the work related with this PhD project in two parts: 
\begin{enumerate*}
\item the first part, focused in the single-robot approach; and
\item the second part, focused in the multi-robot approach.
\end{enumerate*}

Therefore, our future work will be focused in trying to find an answer to the RQ2.
In order to achieve it, a detailed study of the current state of the art of features involved with multi-robot choreography such as emergent properties or collaborative strategies must be performed.
%
%\section{Conclusions}
%\label{sec:conclusion}
%Software engineering can be the key technology needed for the improvement of applications developed for robotic systems.
Robots are nowadays a trend, but without well-defined engineering process for the researchers to follow most of the projects are started from scratch and not time neither cost efficient.
In this project we aim to tackle those problems while validating the premises and results with real-world scenarios where real robots work in a collaborative way.

Furthermore, we plan to perform a detailed research in the state of the art

\balance

\bibliographystyle{ACM-Reference-Format}
\bibliography{sigproc} 

\end{document}
