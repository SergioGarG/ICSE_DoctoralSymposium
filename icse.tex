\documentclass[sigconf,review, anonymous]{acmart}
\acmConference[ICSE 2018]{40th International Conference on Software Engineering}{May 27--June 3, 2018}{Gothenburg, Sweden}
\acmYear{2018}

\usepackage{booktabs} % For formal tables
\usepackage{centernot}
\usepackage{algorithm}
\usepackage{algorithmic}
\usepackage{amsmath,amssymb,amsfonts}
\usepackage{balance}

\newtheorem{remark}{Remark}
\newtheorem{problem}{Problem}
\newtheorem*{running*}{Running example}
\usepackage[inline]{enumitem}
\settopmatter{printfolios=true}


\usepackage[xcolor=orange]{changes}
\definechangesauthor[name={Claudio Menghi},color=orange]{CM}
\definechangesauthor[name={Sergio Garcia},color=red]{SG}
%\definechangesauthor[name={Patrizio},color=blue]{PP}
\usepackage[colorinlistoftodos,prependcaption,textsize=tiny]{todonotes}
%\newcommand{\sergio}[1]{\todo[color=blue]{\textsf{SG} #1}}

\newboolean{showcomments}
\setboolean{showcomments}{true} % toggle to show or hide comments
\ifthenelse{\boolean{showcomments}}
{\newcommand{\nb}[2]{
  \fcolorbox{black}{yellow}{\bfseries\sffamily\scriptsize#1}
  {\sf\small$\blacktriangleright$\textit{#2}$\blacktriangleleft$}
 }
 \newcommand{\version}{\emph{\scriptsize$-$working$-$}}
}
{\newcommand{\nb}[2]{}
 \newcommand{\version}{}
}
\newcommand\patrizio[1]{\nb{Patrizio}{#1}}
\newcommand\claudio[1]{\nb{Claudio}{#1}}
\newcommand\sergio[1]{\nb{Sergio}{#1}}
\newcommand\jana[1]{\nb{Jana}{#1}}

\newcommand{\Sync}{\ensuremath{Meet}}
\newcommand{\cla}[1]{\textcolor{red}{{#1}}}
%%%shortcuts
\newcommand{\robotindex}{n}
\newcommand{\robotindexp}{m}
\newcommand{\setrobot}{\ensuremath{\mathcal{R}}}
\newcommand{\T}{\ensuremath{r}} %transition system

\newcommand{\powerset}{\raisebox{.15\baselineskip}{\Large\ensuremath{\wp}}}
\newcommand{\robot}{\ensuremath{\T}}

\newcommand{\INPUT}{\textbf{Input} }
\newcommand{\OUTPUT}{\textbf{Output} }

\newcommand{\AP}{\Pi} %atomic propositions
\newcommand{\service}{\pi}
\newcommand{\behavior}{\ensuremath{\mathcal{B}}} %Buchi automaton
\newcommand{\trace}{\ensuremath{\mathcal{T}}}
\newcommand{\A}{\ensuremath{\mathcal{A}}} %automaton
\newcommand{\B}{\ensuremath{\mathcal{B}}} %Buchi automaton
\newcommand{\BA}{B\"uchi automaton }
\renewcommand{\P}{\mathcal{P}} %product automaton
\newcommand{\R}{\mathcal{R}} %rabin automaton
\newcommand{\init}{\ensuremath{init}}
\newcommand{\pref}{\mathit{pref}}
\newcommand{\currs}{\mathfrak{s}}
\newcommand{\currq}{\mathfrak{q}}
\newcommand{\dur}{\Delta}
\newcommand{\wait}{\mathit{wait}}
\newcommand{\sync}{\mathit{sync}}
\newcommand{\nosync}{\mathit{nosync}}
\newcommand{\TS}{\ensuremath{\T=(S,} \ensuremath{\init}, \ensuremath{{A},} \ensuremath{\AP,} \ensuremath{T,} \ensuremath{\Sync,} \ensuremath{L)}}
\newcommand{\PTS}{\ensuremath{\T=(S,} \ensuremath{\init},   \ensuremath{{A},} \ensuremath{\AP,}    \ensuremath{T,}   \ensuremath{T_p,}  \ensuremath{\Sync}, \ensuremath{\Sync_p}, \ensuremath{L} )}
\newcommand{\PTSone}{\ensuremath{\T'=(S',} \ensuremath{\init',} \ensuremath{{A'},} \ensuremath{\AP',} \ensuremath{T',}  \ensuremath{T'_p,} \ensuremath{\Sync' )}}

\newcommand{\ra}{$\rightarrow$}
\newcommand{\ugh}[1]{\textcolor{red}{\uwave{#1}}} % please rephrase
\newcommand{\ins}[1]{\textcolor{blue}{\uline{#1}}} % please insert
\newcommand{\del}[1]{\textcolor{red}{\sout{#1}}} % please delete
\newcommand{\chg}[2]{\textcolor{red}{\sout{#1}}{\ra}\textcolor{blue}{\uline{#2}}}

\newcommand{\TSprime}{{\T'=(S',\init',{A}',T', \Sync')}}

\newcommand{\TSoneprime}{{\T_1=(S'_1, \init'_{1},{A}'_1,T'_1, \Sync')}}

\newcommand{\TSone}{{\T_1=(S_1,\init_{1},{A}_1,T_1, \Sync_1)}}
\newcommand{\TStwo}{{\T_1=(S_2, \init_{2},{A}_2,T_2, \Sync_2)}}
\newcommand{\TSproduct}{{(S_1 \times S_2,\langle \init_{1}, s_{2,\init} \rangle,{A}_2 \cup {A}_2,T, \Sync)}}
\newcommand{\TStwoprime}{{\T_1=(S'_2,\init'_{2},{A}'_2,T'_2)}}
\newcommand{\robotpar}{\ensuremath{\T_\robotindex=} \ensuremath{(S_\robotindex,} \ensuremath{\init_{\robotindex},} \ensuremath{{A}_\robotindex,} \ensuremath{\AP_\robotindex ,} \ensuremath{T_\robotindex,} \ensuremath{\Sync_\robotindex)}}
\newcommand{\robotp}[1]{\ensuremath{\T_{#1}=} \ensuremath{(S_{#1},} \ensuremath{\init_{#1},} \ensuremath{{A}_{#1},} \ensuremath{\AP_{#1} ,} \ensuremath{T_{#1},} \ensuremath{\Sync_{#1})}}
\newcommand{\partialrobotp}[1]{\ensuremath{\T_{#1}}=(\ensuremath{S_{#1}}, \ensuremath{\init_{#1},} \ensuremath{{A}_{#1},} \ensuremath{\AP_{#1},} \ensuremath{T_{#1},} \ensuremath{T_{p,#1},} \ensuremath{\Sync_{#1},} 
\ensuremath{\Sync_{p,#1}}, \ensuremath{{L}_{#1}})}


\newcommand{\PTSpar}{\ensuremath{\T_\robotindex=(S_\robotindex,} \ensuremath{\init_{\robotindex},} \ensuremath{{A}_\robotindex,} \ensuremath{\AP_\robotindex ,} \ensuremath{T_\robotindex,} \ensuremath{T_{p,\robotindex},} \ensuremath{\Sync_\robotindex}, 
\ensuremath{\Sync_{p,\robotindex}, \ensuremath{L}_\robotindex)}}

\newcommand{\network}{\ensuremath{\N}}
\newcommand{\robotapplication}{\ensuremath{\N}}
\newcommand{\robotset}{\network = \{\ensuremath{\robot_1,} \ensuremath{\robot_2, \ldots}, \ensuremath{\robot_N\}}}
\newcommand{\robotsetdef}{\{\ensuremath{\robot_1,} \ensuremath{\robot_2, \ldots}, \ensuremath{\robot_N\}}}
\newcommand{\robotsetdefp}{\{\ensuremath{\robot'_1,} \ensuremath{\robot'_2, \ldots}, \ensuremath{\robot'_N\}}}
\newcommand{\property}{\ensuremath{\phi}}
\newcommand{\missions}{\ensuremath{\Phi}}
\newcommand{\propertiesset}{\ensuremath{\Phi =} \{\ensuremath{\property_1,} \ensuremath{\property_2, \ldots}, \ensuremath{\property_N\}}}


\newcommand{\TSi}{{\T_i=(S_i,\init_{i},{A_i} ,T_i)}}
\newcommand{\N}{\mathcal{H}}
\newcommand{\M}{{M}}
\newcommand{\model}{\mathcal{M}}
\newcommand{\I}{\mathcal{I}}
\newcommand{\D}{{D}}
\renewcommand{\O}{\mathcal{O}}

\newcommand{\toolName}{MAPmAKER}
\newcommand{\APs}{\mathbf{\Pi}}
\newcommand{\Lang}{\mathcal{L}} %language
\newcommand{\Set}{\mathsf{S}} %set
\newcommand{\Spec}{\mathbf{\Phi}}
\newcommand{\Epsilon}{\mathcal{E}}
\renewcommand{\i}{\iota}
\newcommand{\Nat}{\mathbb{N}} %natural numbers
\newcommand{\Real}{\mathbb{R}}
\newcommand{\Next}{\mathsf{X}}
\newcommand{\Until}{\mathsf{U}}
\newcommand{\Always}{\mathsf{G}}
\newcommand{\Event}{\mathsf{F}}
\newcommand{\false}{\mathit{false}}
\newcommand{\true}{\mathit{true}}
\newcommand{\trueval}{\ensuremath{\top}}
\newcommand{\falseval}{\ensuremath{\bot}}
\newcommand{\maybe}{\ensuremath{?}}
\renewcommand{\epsilon}{\varepsilon}
\newcommand{\prop}{\pi}
\newcommand{\ie}{{i.e., }}
\newcommand{\eg}{{e.g., }}
\newcommand{\progressive}{\varpi}
\newcommand{\move}{\mathit{move}}
\newcommand{\h}{h}
\renewcommand{\H}{H}
\newcommand{\parti}{\mathit{P}}
\newcommand{\Alpha}{\mathbf{\Sigma}}
\renewcommand{\mod}{\mathrm{\, mod \, }}
\newcommand{\suc}{\mathit{succ}}
\newcommand{\dist}{\mathrm{dist}}
\newcommand{\proj}{\mathrm{proj}}
\newcommand{\parspace}{\vskip 0.05in}


\makeatletter
\newcommand*{\da@rightarrow}{\mathchar"0\hexnumber@\symAMSa 4B }
\newcommand*{\da@leftarrow}{\mathchar"0\hexnumber@\symAMSa 4C }
\newcommand*{\xdashrightarrow}[2][]{%
  \mathrel{%
    \mathpalette{\da@xarrow{#1}{#2}{}\da@rightarrow{\,}{}}{}%
  }%
}
\newcommand{\xdashleftarrow}[2][]{%
  \mathrel{%
    \mathpalette{\da@xarrow{#1}{#2}\da@leftarrow{}{}{\,}}{}%
  }%
}
\newcommand*{\da@xarrow}[7]{%
  % #1: below
  % #2: above
  % #3: arrow left
  % #4: arrow right
  % #5: space left 
  % #6: space right
  % #7: math style 
  \sbox0{$\ifx#7\scriptstyle\scriptscriptstyle\else\scriptstyle\fi#5#1#6\m@th$}%
  \sbox2{$\ifx#7\scriptstyle\scriptscriptstyle\else\scriptstyle\fi#5#2#6\m@th$}%
  \sbox4{$#7\dabar@\m@th$}%
  \dimen@=\wd0 %
  \ifdim\wd2 >\dimen@
    \dimen@=\wd2 %   
  \fi
  \count@=2 %
  \def\da@bars{\dabar@\dabar@}%
  \@whiledim\count@\wd4<\dimen@\do{%
    \advance\count@\@ne
    \expandafter\def\expandafter\da@bars\expandafter{%
    }%
  }%  
  \mathrel{#3}%
  \mathrel{%   
    \mathop{\da@bars}\limits
    \ifx\\#1\\%
    \else
      _{\copy0}%
    \fi
    \ifx\\#2\\%
    \else
      ^{\copy2}%
    \fi
  }%   
  \mathrel{#4}%
}

\everymath{\vadjust{\nobreak\null}}

\makeatletter
\def\old@comma{,}
\catcode`\,=13
\def,{%
  \ifmmode%
    \old@comma\discretionary{}{}{}%
  \else%
    \old@comma%
  \fi%
}
\makeatother

\def\HiLi{\leavevmode\rlap{\hbox to \hsize{\color{yellow!50}\leaders\hrule height .8\baselineskip depth .5ex\hfill}}}

\newif\ifextended
\extendedfalse
%\extendedtrue

\ifextended
\newcommand{\extended}[1]{\textcolor{red}{#1}} 
\newcommand{\notextended}[1]{}
\else
\newcommand{\extended}[1]{}
\newcommand{\notextended}[1]{#1}
\fi

\begin{document}
\title{MAPmAKER: A Tool for Performing Multi-Robot LTL Planning Under Uncertainty}


%\author{Sergio Garc\'{i}a}
%\affiliation{%
%  \institution{University of Gothenburg}
%  \streetaddress{P.O. Box 1212}
%  \city{Gothenburg} 
%  \state{Sweden} 
%  \postcode{405 30}
%}



\begin{abstract}
Robot applications are increasingly asking for decentralized techniques that allow for tractable automated planning. 
Furthermore, those applications can be deployed in dynamic environments where its uncertainty must be handled.
Typically, environments where human beings are involved can just provide a partial knowledge of its model, i.e. the current state of a door between two rooms in uncertain. 

Our proposed tool, \toolName tackle the limitations that current planning techniques are used for teams of robots: 
\begin{enumerate*}
\item it decomposes the robotic team into subclasses, avoiding the not scalable centralized approach;
\item it considers complex-high level missions given in temporal logic;
\item it is able to work also with only partial knowledge of the environment, performing possible plans.
\end{enumerate*}

\end{abstract}

%
% The code below should be generated by the tool at
% http://dl.acm.org/ccs.cfm
% Please copy and paste the code instead of the example below. 
%
%\begin{CCSXML}
%<ccs2012>
% <concept>
%  <concept_id>10010520.10010553.10010562</concept_id>
%  <concept_desc>Computer systems organization~Embedded systems</concept_desc>
%  <concept_significance>500</concept_significance>
% </concept>
% <concept>
%  <concept_id>10010520.10010575.10010755</concept_id>
%  <concept_desc>Computer systems organization~Redundancy</concept_desc>
%  <concept_significance>300</concept_significance>
% </concept>
% <concept>
%  <concept_id>10010520.10010553.10010554</concept_id>
%  <concept_desc>Computer systems organization~Robotics</concept_desc>
%  <concept_significance>100</concept_significance>
% </concept>
% <concept>
%  <concept_id>10003033.10003083.10003095</concept_id>
%  <concept_desc>Networks~Network reliability</concept_desc>
%  <concept_significance>100</concept_significance>
% </concept>
%</ccs2012>  
%\end{CCSXML}
%
%\ccsdesc[500]{Computer systems organization~Embedded systems}
%\ccsdesc[300]{Computer systems organization~Redundancy}
%\ccsdesc{Computer systems organization~Robotics}
%\ccsdesc[100]{Networks~Network reliability}


%\keywords{ACM proceedings, \LaTeX, text tagging}


\maketitle

\section{Introduction}
\label{sec:introduction}
%Helping designers to engineer robotic systems is one of the timely application areas of software engineering. 
%Designing %good 
%robotic systems requires solving %several 
%numerous problems~\cite{ljungblad2005designing}, such as the selection of the types of the robot to be employed (e.g., humanoid, robotic arm), the analysis of the required robot properties (e.g., emergence, emotional), the analysis of the place in which the robot operates (e.g., on the bus, in a birthday party), the activity the robot has to perform (e.g., move object, reach a location), and the users it has to serve (e.g., taxi driver, rock star). 
%These aspects are then used by designers in the selection of appropriate planners.

\toolName provides a planner where a robot application is defined using finite transition systems.
A \emph{planner} is  a software component that receives as input a model of the robotic application and computes  a set of actions (a \emph{plan}) that, if performed, allows the achievement of a desired mission~\cite{latombe2012robot}.
Each robot application contains the robots that conform the team and the mission that they have to achieve.

A \emph{global mission} represents the high-level mission that must be accomplished by the whole team \cite{kloetzer2011multi,loizou2005automated,quottrup2004multi} and that is decomposed into a set of \emph{local missions}\cite{schillinger2016decomposition,guo2015multi,guo2015multi,tumova2016multi}.
Every robot is commanded to achieve a local mission, specified as a LTL property.
As seen in \cite{tumova2016multi}, this collaborative fashion of accomplishing the global mission is performed in a \emph{decentralized} way.
Each robot that is part of a subset of the team computes the solution for its own sub-mission, avoiding the expensive fully centralized planning and making it more robust to local problems.

Nowadays, most of the planners consider the model of the environment as known and not dynamic~\cite{7139412}. 
However, this is not a real condition of real world scenarios, where only \emph{partial knowledge} can be ensured.
For this reason, our tool is able to compute a plan even when only partial information of the environment is available, as seen in \cite{roy2006planning,du2012robot,diaz2001exploring}.
However, the novelty of our work consists in fuse all this features, exploiting a \emph{decentralized} methodology.
This kind of approaches are not yet studied in detail, due to there are only a few planners managing this issues \cite{guo2015multi}.

\textbf{Organization.} 
Section~\ref{sec:limitations} introduces robotic applications by highlighting the status of current planners.
Section~\ref{sec:approach} describes the \toolName\ approach.
Section~\ref{sec:tool} presents the \toolName\ tool.
Section~\ref{sec:conclusion} concludes with final remarks.

%\section{Related work}
%\label{sec:related}
%\emph{Decentralized solutions.}
%Kind of state of the art, relate de tool 

\sergio{Not sure if we should add some planner tools as V-Rep http://www.coppeliarobotics.com/, ROS Moveit http://moveit.ros.org/, Gazebo http://gazebosim.org/ or look for papers presenting tools}

Decentralized planning problem has been studied for known environments~\cite{schillinger2016decomposition,guo2015multi,tumova2016multi}.
However, planners for partially known environments do not usually employ decentralized solutions~\cite{roy2006planning,du2012robot,diaz2001exploring}. 

\emph{Dealing with partial knowledge in planning.}
Planning in partially known environments is handled in different ways. 
\begin{enumerate*}
\item Several works (e.g.,~\cite{ding2011ltl,kurniawati2011motion,wolff2012robust,du2012robot,Roy2006,chen2012ltl,nikou2017probabilistic,7078886,7139350,narayanan2015task}) consider probabilities within the planning algorithm.
Most of these works  treat partial information by modeling the robotic application using some form of \emph{Markov decision processes} (MDP).
In some of these works~\cite{ding2011ltl,chen2012ltl} transitions of the robots are associated with probabilities which indicate the probability of reaching the destination of the transition given that an action is performed.
In other works~\cite{wolff2012robust}, transition probabilities are not exactly known but are known to belong to a given uncertainty sets.
Finally, several works~\cite{kurniawati2011motion,Roy2006} consider partially observable Markov decision processes.
All these approaches generally generate plans that maximize the worst-case probability of satisfying a mission.
Differently, our work does not consider probabilities.
\item Several works (e.g,~\cite{lahijanian2016iterative,livingston2012backtracking,l2014safety,nie2016searching,7139412}) studied how to change the planned trajectories when unknown obstacles are detected or when obstacles move in a unpredictable way.
In this case, the used underlying model is some sort of \emph{hybrid model}, i.e., models in which finite state machines are combined with differential equations. 
In~\cite{lahijanian2016iterative}, to plan trajectories the authors use a high-level planner that exploits an abstraction of the hybrid system and the mission to compute high-level plans. 
The low-level planner uses the dynamics of the hybrid system and the suggested high-level plans to explore the state space for feasible solutions.
Every time an unknown obstacle is encountered, the high-level planner modifies the coarse high-level plan online by accounting for the geometry of the discovered obstacle. 
Within this framework, \toolName\ can be considered as a high-level planner that is able to use an abstraction of the hybrid system that contains partial information, i.e., encode unknown obstacles.
\item Some  approaches  analyzed how to update plans when new information about known model of a robotic application is detected (e.g.,~\cite{guo2015multi}). 
Differently, in our approach portions of the model of the robotic application are partially known,  partial knowledge is reduced as true and false evidence about partial information is detected.
Other works (e.g.,~\cite{7139310}), aim at detecting how to explore totally unknown environments.
\item 
Plan synthesis is a particular instance of controller synthesis. 
Controller  synthesis (e.g.,~\cite{cassandras2009introduction,D'ippolito:2013:SNE:2430536.2430543}) aims at finding a component, usually indicated as controller or supervisor, that ensures property satisfaction for all the possible system executions.
Differently, plan synthesis aims at finding a single execution, i.e., a plan that ensures property satisfaction.
The controller synthesis  is usually (\cite{kress2009temporal,wongpiromsarn2009receding,chen2012ltl,livingston2012backtracking,guo2013revising}) performed by solving a two player game between robots and their environment.
The goal is to find a strategy the robots can use that allows always winning the game.
Differently, in our case the planning algorithm ensures that there is a way of completing the \emph{single} (possible) plan that satisfies the property of interest. 
\item \toolName\ can be classified on the boundary between reactive synthesis~\cite{chen2012ltl,livingston2012backtracking,thomas2002automata} techniques and iterative planning~\cite{guo2013revising,maly2013iterative}. 
As reactive synthesis techniques, \toolName\ constructs a control strategy that accounts for every possible variation in the environment, but the computed plan does not allow always winning the  \emph{two player game} between the robots and their environment.
As  iterative planning, a new plan is computed on-the-fly when new information is available.
\end{enumerate*}

%
%\section{Research approach}
%\label{sec:approach}
%In this project we are working from the academic point of view research-wise but we are also working closely with the industry.
As stated before, the work developed for this PhD project is embedded in the Co4Robots European project.
The main goal of this project is to deploy a robotic application in a ``domestic" environment such as hospitals, hotels, airports, etc.
These robotic applications must be able to accomplish complex missions with a systematic, real-time, decentralized methodology.
Furthermore, a robotic application could be composed by a team of potentially heterogeneous robots.
The aforementioned environments will be considered as dynamic and will also count with the presence of human beings.
For this reason, the robots must have integrated a set of perceptual capabilities that enables them to localize themselves and estimate the state of their highly dynamic environment in the presence of strong interactions and in a collaborative manner.
Robots must not only interact between them, but also with human beings.

In order to validate the code and artifacts developed for Co4Robots, the committee defined a set of study cases for the project proposal.
Our framework builds upon various cases of base interactions between agent pairs of different types. 
The considered inter-agent interactions are:
\begin{itemize}
\item[Case A] physical guidance by a human for the transportation of an object carried by a robot;
\item[Case B] collaborative grasping and manipulation of an object by two agents;
\item[Case C]collaborating mobile platform and stationary manipulator to facilitate loading and unloading tasks onto the mobile platform; and
\item[Case D] information exchange between a human giving orders and a robotic agent. 
\end{itemize}

On the other hand, it is important to decouple the research made just for the project and the one intended for the whole PhD.
While the single-robot part can be more driven by the Co4Robots goal, the second part is expected to reach far beyond results than the intended for the project.
For example, the multi-robot choreography for Co4Robots is limited by its study cases to two robots.
However, for our own research we plan to manage a team of them acting under some defined collaborative strategies and adapting to emergent properties.


%
%\section{Current status}
%\label{sec:current}
%Currently, our main focus lays on the RQ1, learning, defining and differencing between software engineering practices regarding single-robot and multi-robot systems.
Our software architecture, Self-adaptive dEcentralised Robotic Architecture (SERA) is already defined.
As its name indicates, it supports a real-time decentralized robot coordination to accomplish missions with teams of robots. 
Furthermore, it is self-adaptive, responding to different changes by computing new strategies to achieve the desired goals.
SERA was already tested during an Integration Meeting of the project, where it demonstrate that can support the performance of a robot achieving different complex missions ---i.e. collaborative transportation with an human being, autonomous driving in a dynamic environment.

The aforementioned architecture is three layers architecture that is strongly influenced by the well-known work of Kramer and Magee~\cite{kramer}.
It has the same structure, but, as depicted in Figure~\ref{fig:arch}, we also added a new item that works as a central station.
It is important to remark that the aim of our project is to build a system that can be easily used by not technical users, so we had to define a way for them to command the missions to the robotic team.
The central station is just used during design-time in order to allocate a graphical interface to be used by a final user.

On the other hand, SERA follows the component-based style, so the main robotic functionalities are encapsulated in different modules or "components".
All this components are developed abstracting the communication capabilities since we rely on the interfaces defined in the architecture.
It not only significantly reduces the complexity of the code but also triggers the modularity of our system making possible exchanging the components that conform our architecture.

\begin{figure*}[!t]
\begin{center}
\includegraphics[width=1\linewidth]{Figures/InstanceMultiRobot_Graffle.pdf}
\caption{Software architecture.}
\label{fig:arch}
\end{center}
\end{figure*}

Since the components of our architecture are exchangeable our next short-term goal is to define configuration facilities that can be applied to our system depicted in the architecture.
It will allow to our applications to support two things:
\begin{enumerate}
\item Being customizable at design-time, so we can configure its components based on the requirements of our context (i.e. hardware installed in each robot, environment where they will be deployed, etc.)
\item To self-adapt or self-configure at run time, so each robot can apply changes in its configuration based on emergent events of the environment or failures of their system.
\end{enumerate}

In order to do so we will implement pluginlib~\footnote{http://wiki.ros.org/pluginlib}, a package that uses the ROS build infrastructure and provides tools for writing and dynamically loading plugins.

Finally, in order to communicate each robot with its teammates we implemented an approach based on ROS+REST.
So, using a suitable component that works as an interface we are able to send messages in form of services between robots.
In this way, each robot has an instance of ROS running in their own local environment so we can deploy a whole team of robots avoiding a central master node and the problems related with this approach (i.e. bottleneck issues, less robustness facing failures of a node, etc.), specially working with the ROS middleware.


%
%\section{Future work and directions}
%\label{sec:future}
%As stated before, we plan to split the work related with this PhD project in two parts: 
\begin{enumerate*}
\item the first part, focused in the single-robot approach;
\item and the second part, focused in the multi-robot approach.
\end{enumerate*}

Therefore, our future work will be focused in trying to find an answer to the RQ2.
In order to achieve it, a detailed study of the current state of the art of features involved with multi-robot choreography such as emergent properties or collaborative strategies must be performed.
%
%\section{Conclusions}
%\label{sec:conclusion}
%Software engineering can be the key technology needed for the improvement of applications developed for robotic systems.
Robots are nowadays a trend, but without well-defined engineering process for the researchers to follow most of the projects are started from scratch and not time neither cost efficient.
In this project we aim to tackle those problems while validating the premises and results with real-world scenarios where real robots work in a collaborative way.

Furthermore, we plan to perform a detailed research in the state of the art

\balance

\bibliographystyle{ACM-Reference-Format}
\bibliography{sigproc} 

\end{document}
