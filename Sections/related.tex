In this section we discuss related work with respect with our two proposed Research Questions.

Our robotic application will be based on ROS~\cite{Quigley2009} and some functionalities will be build on top of it. 
ROS is an open-source meta-operating system for robots. 
It provides a communication layer above the Linux host operating system that supports the execution of components in a distributed system. 
ROS offers message-based peer-to-peer communication infrastructure supporting the integration of independently developed software components, called ROS nodes, that are organized into a graph.

The benefits for using ROS are many and one of them is the flexibility that this tool provides to the developers.
However, this flexibility could result in a development process based on ad-hoc solutions rather than being based on a systematic engineered approach. 
Obviously, it decreases the modularity and reusability of the developed system and makes its development process to take longer due to many of its applications will have to be generated from scratch. 
Furthermore, ROS has some limitations, some of them recognized by their developers~\footnote{http://design.ros2.org/articles/why\_ros2.html}.
ROS2 is supposed to solve these previous problems and to substitute ROS1 in a near future, but since it was just released and there are not yet all the contents that were available for ROS1 we opted for keep using ROS1.

Architectures
	Decentralized
	Multi-agent
	Not SOA, microservices, cloud container

Collaborative adaptation
