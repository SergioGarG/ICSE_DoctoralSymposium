--Explain the tool in detail, maybe including a scope-- (add figure for the tool? maybe in the previous section?)

Our tool was developed using the MATLAB~\cite{matlab} environment.
The tool is composed by several MATLAB-based scripts that can be executed independently for performing certain functions.
The usage of whole tool is launched using a function where the \emph{research question}, the \emph{example} or map of the environment and the \emph{experiment} to check are their arguments.
his function enormously simplifies the performance of the tool, but there are other functionalities that can also be exploited explained in the previous section.

In a normal case where the scenarios are already defined, the only components of the architecture depicted in~\ref{fig:overview} that are used are the \emph{Replication Package}, the \emph{Robot \& Environment models with uncertainty}, the \emph{Planner} itself and its inputs the \emph{Solutions container}.
The first component represent a folder where the models of the robot and environment with a certain added uncertainty for the selected experiment are stored.
They are loaded and set as MATLAB workspace variables in the second component.
The planner needs as input the information from this variables and also from the scripts \emph{Algorithms}, \emph{Utils} and \emph{Configuration files}, that are automatically linked.
Once the experiment is concluded, the data extracted (as explained in~\ref{sec:approach}) is stored in the solutions container component.

\sergio{TODO: explain generation of models and fix diagram}

\textbf{Performance}

In \sergio{cite previous work} we have shown that our tool is able to perform a plan even in environments full of uncertainty.
Thus, \toolName is able to compute a plan where other planners will not.
However, the computation time and the number of actions to be performed by the robot that tries to follow a possible plan and detects a false evidence are normally greatly increased.
This increasing is due to required re-computing that has to be performed when looking for a new possible plan.

For checking its performance the tool was tested in different environments based on real-world scenarios.
The model of the uncertainty of the environment was randomly created and matched with randomly models of the robots (service and synchronization locations and uncertainties).
The performance of tool was evaluated in two Research Questions divided in three experiments (one for each kind of uncertainty defined for this project).
The results are presented and discussed in \sergio{cite previous work}

\textbf{Tool Validation}

The \toolName is available at ~\url{https://goo.gl/GY7ZrG}.
There, the following features can be found:
\begin{enumerate}
\item a complete replication package of the two RQs and experiments proposed at \sergio{cite other work},
\item  a set of videos showing \toolName\ in action computing and solving the scenarios presented at the previous bullet,
\item a brief user guide which defines the functionalities provided by our tool.
\end{enumerate}
