The plan within this project is to split the work in two parts:
\begin{enumerate*}
\item the first part, focused on the study of current practices for engineering robotic applications (RQ1) and the development of the platform allocated within each robot (RQ2); and
\item the second part, focused on collaboratively managing a team of robots (RQ3).
\end{enumerate*}

This project is being developed closely with two companies\footnote{https://www.bosch.com/}\footnote{https://pal-robotics.com/en/home/} that are partners of the Co4Robots European project.
Its main goal is to deploy a robotic application in ``domestic" environments such as hospitals, hotels, airports, etc.
These environments will be considered as dynamic ---i.e. changing environments with presence of uncertainty--- and with presence of human beings.
We consider that robotic applications must be able to accomplish complex missions with a systematic, real-time, and decentralized methodology.
%Furthermore, a robotic application could be composed by a team of potentially heterogeneous robots.
For this reason, the robots must have integrated a set of perceptual capabilities that enable them to localize themselves and estimate the state of their highly dynamic environment in the presence of strong interactions and in a collaborative manner.
That is, robots must not only interact among them, but also with human beings.

In order to learn which are the current practices for developing robotic applications we performed an extensive research in the field.
We also plan to conduct empirical studies with different companies, starting with the industrial partners of Co4Robots.
With this steps we expect to answer RQ1. %and also to figure out which are the limitations of such practices.  

%A not negligible task within this project is to decouple the research made just for Co4Robots and the one intended for the whole PhD.
%While the platform development outcomes are common for both our own research and the Co4Robots' , our outcomes of the collaborative adaptation part are expected to reach far beyond results than the intended by Co4Robots.
%For example, for our own research we plan to manage a team of robots acting under some defined collaborative strategies and adapting to emergent properties.



