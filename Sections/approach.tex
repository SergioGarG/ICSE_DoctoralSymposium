The plan within this project is to split the work time and topic wise done  in two parts
\begin{enumerate*}
\item the first part, focused on the study of current practices for engineering robotic applications (RQ1) and the development of the framework allocated within each robot (RQ2); and
\item the second part, focused on collaboratively managing a team of robots (RQ3).
\end{enumerate*}

In this project we are working from the academic point of view but we are also working closely with the industry.
As stated before, the work developed for this PhD project is embedded in the Co4Robots European project.
The main goal of this project is to deploy a robotic application in a ``domestic" environment such as hospitals, hotels, airports, etc.
These robotic applications must be able to accomplish complex missions with a systematic, real-time, decentralized methodology.
Furthermore, a robotic application could be composed by a team of potentially heterogeneous robots.
The aforementioned environments will be considered as dynamic and will also count with the presence of human beings.
For this reason, the robots must have integrated a set of perceptual capabilities that enables them to localize themselves and estimate the state of their highly dynamic environment in the presence of strong interactions and in a collaborative manner.
Robots must not only interact between them, but also with human beings.

In order to learn which are the current practices for developing robotic applications we performed an extensive research in the field.
We also plan to conduct empirical studies such as interviews with different companies, starting with the industrial partners of Co4Robots.
With this steps we expect to answer RQ1 and also to figure out which are the limitations of such practices.  

A not negligible task within this project is to decouple the research made just for Co4Robots and the one intended for the whole PhD.
While the single-robot part expected outcomes of both our own research and the Co4Robots' are the same, our outcomes of the collaborative adaptation part are expected to reach far beyond results than the intended by Co4Robots.
For example, for our own research we plan to manage a team of them acting under some defined collaborative strategies and adapting to emergent properties.



