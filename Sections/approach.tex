In this project we are working from the academic point of view research-wise but we are also working closely with the industry.
As stated before, the work developed for this PhD project is embedded in the Co4Robots European project.
The main goal of this project is to deploy a robotic application in a ``domestic" environment such as hospitals, hotels, airports, etc.
These robotic applications must be able to accomplish complex missions with a systematic, real-time, decentralized methodology.
Furthermore, a robotic application could be composed by a team of potentially heterogeneous robots.
The aforementioned environments will be considered as dynamic and will also count with the presence of human beings.
For this reason, the robots must have integrated a set of perceptual capabilities that enables them to localize themselves and estimate the state of their highly dynamic environment in the presence of strong interactions and in a collaborative manner.
Robots must not only interact between them, but also with human beings.

In order to validate the code and artifacts developed for Co4Robots, the committee defined a set of study cases for the project proposal.
Our framework builds upon various cases of base interactions between agent pairs of different types. 
The considered inter-agent interactions are:
\begin{itemize}
\item[Case A] physical guidance by a human for the transportation of an object carried by a robot;
\item[Case B] collaborative grasping and manipulation of an object by two agents;
\item[Case C]collaborating mobile platform and stationary manipulator to facilitate loading and unloading tasks onto the mobile platform; and
\item[Case D] information exchange between a human giving orders and a robotic agent. 
\end{itemize}

Therefore, one the most important ways of testing and validating of most of the results obtained in this research are the previously stated study cases.
For each study case, the Co4Robots consortium holds a \emph{Milestone meeting} in order to test all the developed tools.
It allow us not only to test our research outcomes in real robots working in real-world scenarios but also to discuss with the rest of the developing stakeholders involved in the project and collecting valuable feedback.
The outline of the approach that we are currently follow for our approach and the validation is as follows:
\begin{enumerate}
\item Personal work alternated with internal meetings.
\item Presentation of achieved work to the Co4Robots committee and collection of feedback.
\item Correction of work based on feedback.
\item Validation during Milestones and Integration Meetings.
\item Documentation for deliverables requested for every task within each Workpackage of the project.
\end{enumerate}

On the other hand, it is important to decouple the research made just for the project and the one intended for the whole PhD.
While the single-robot part expected outcomes of both our own research and the Co4Robots' are the same, our outcomes of the second part are expected to reach far beyond results than the intended by Co4Robots.
For example, the multi-robot choreography for Co4Robots is limited by its study cases to two robots.
However, for our own research we plan to manage a team of them acting under some defined collaborative strategies and adapting to emergent properties.



