%In order to validate the code and artifacts developed for Co4Robots, the consortium defined a set of study cases in the project proposal.
%Our framework builds upon various cases of base interactions between agent pairs of different types. 
%The considered inter-agent interactions are:
%\begin{itemize}
%\item[Case A] physical guidance by a human for the transportation of an object carried by a robot;
%\item[Case B] collaborative grasping and manipulation of an object by two agents;
%\item[Case C]collaborating mobile platform and stationary manipulator to facilitate loading and unloading tasks onto the mobile platform; and
%\item[Case D] information exchange between a human giving orders and a robotic agent. 
%\end{itemize}

In order to validate the code and artifacts developed for Co4Robots we defined three different approaches than can be followed depending on the artifact that we want to obtain.
The first approach consists in presentations of our work during meetings and consortiums and get feedback from stakeholders and practitioners.
For the second we make use of simulation tools that allow us to validate our artifacts before implementing them into real robots.
Finally, the third one consists in validation with real robots in real-world scenarios.
For example, the outline of the approach that we followed for validating SERA is:
\begin{enumerate}
\item Personal work alternated with internal meetings.
\item First validation by means of simulating scenarios.
\item Presentation of achieved work to the Co4Robots committee and collection of feedback.
\item Correction of work based on feedback.
\item Validation during Milestones and Integration Meetings with real robots in real-world scenarios.
\item Documentation for deliverables requested for every task within each Workpackage of the project.
\end{enumerate}


%The case studies are the best tool for testing and validating most of the results obtained in this research.
%For each case study, the Co4Robots consortium holds a \emph{Milestone meeting} in order to test all the developed tools.
%It allow us not only to test our research outcomes in real robots working in real-world scenarios but also to discuss with the rest of the developing stakeholders involved in the project and to collect valuable feedback.

%The outline of the approach that we are currently follow for validation of our research is as follows:

%In this case we combined the three stated validation approaches.
%However, there are other cases ---such as the validation of the collaborative adaptation algorithms--- where we plan to validate our artifacts starting from the second approach and then following with the third one since they do not need validation from the Co4Robots stakeholders.





