In order to validate the code and artifacts developed for Co4Robots, the committee defined a set of study cases in the project proposal.
Our framework builds upon various cases of base interactions between agent pairs of different types. 
The considered inter-agent interactions are:
\begin{itemize}
\item[Case A] physical guidance by a human for the transportation of an object carried by a robot;
\item[Case B] collaborative grasping and manipulation of an object by two agents;
\item[Case C]collaborating mobile platform and stationary manipulator to facilitate loading and unloading tasks onto the mobile platform; and
\item[Case D] information exchange between a human giving orders and a robotic agent. 
\end{itemize}

The study cases are the best tool for testing and validating most of the results obtained in this research.
For each study case, the Co4Robots consortium holds a \emph{Milestone meeting} in order to test all the developed tools.
It allow us not only to test our research outcomes in real robots working in real-world scenarios but also to discuss with the rest of the developing stakeholders involved in the project and to collect valuable feedback.
The outline of the approach that we are currently follow for validation of our research is as follows:
\begin{enumerate}
\item Personal work alternated with internal meetings.
\item First validation by means of simulating scenarios.
\item Presentation of achieved work to the Co4Robots committee and collection of feedback.
\item Correction of work based on feedback.
\item Validation during Milestones and Integration Meetings.
\item Documentation for deliverables requested for every task within each Workpackage of the project.
\end{enumerate}





