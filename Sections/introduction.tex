Service robots are increasingly being involved in human lives. 
They are increasingly used in environments such as houses, airports, hospitals, and offices for performing navigation, transportation, and manipulation tasks. 
The World Robotic Survey~\cite{wrs:online} estimated 35 million indoor service robots to be sold by 2018, accumulating a sales value of \$12 billion since 2015. 
The global sales of household and personal robots is expected to grow by 23.5\% per year~\cite{sheng:online}. 
This increase is accompanied with huge progress in robot technology, especially in image processing, planning, control, and collaboration. Software engineering is key to sustaining this new technology.

A robot typically performs specialized tasks; however, some tasks are highly complex and require a team of robots, whose capabilities (e.g., perception, manipulation, and actuation) are coordinated and supervised. 
Such teams also need to adapt to changes, such as of the environment, of the desired tasks, or of the robot (e.g., hardware failures). 
These demands drive the complexity of robot control software relying on appropriate software architectures. 
To tackle this complexity, we need to rethink design processes~\cite{Lee2008} by properly managing system integration and raising the abstraction levels, addressing qualities like evolvability~\cite{Perez2008}, configurability~\cite{Gamez2013563}, scalability and dependability.

The PhD project presented in this paper is involved in the Co4Robots~\footnote{http://www.co4robots.eu/} European project.
Being part of this project allow us not only to formulate but also to validate our research questions in real-world scenarios.
According to~\cite{roadmap}: ``Usually there are no system development processes (highlighted by a lack of overall architectural models and methods). 
This results in the need for craftsmanship in building robotic systems instead of following established engineering processes".
In fact, most of the robotic applications developed nowadays have to be started from scratch without following well-engineered methods to help in the process.
In this context the aim of the Co4Robots project is to establish systematic engineering process to facilitate the development of the software for animating robotic systems through the creation of reusable robot building blocks with well-defined interfaces and properties. 

The research conducted during the present project will be split in two parts.
The first one defines the best practices for engineering software robotic applications.
During this period we studied the current software engineering practices for both single and multi-robotic systems.
Furthermore, throughout this part of the research we will develop the platform that will be allocated within every robot of our applications.
An instance of our \emph{Software platform} will be deployed on each robot.
It will integrate the \emph{Software architecture} of the whole system and  the \emph{Configuration facilities}, which provide the required tools for configuring our architecture both at design and run-time.

The second part of the research aims to support the \emph{choreography} of robotic applications.
It is considered future work that will be tackled once the platform is completely addressed.
In this context, choreography means the way of representing and controlling the interactions between multiple services of a system in a decentralized way.
The main expected outcome of this part is to perform the choreography of a deployed team of potentially heterogeneous robots in dynamic environments with the presence of human beings.
In order to do so, issues as \emph{Emergent properties} ~\cite{DeAngelis2015,DeAngelis2016} and selecting the most suitable \emph{Collaborative adaptation}~\cite{Yan2013} techniques must be addressed.

\textbf{Research Questions.} 
Based on the division of the project, we state the following research questions:
\begin{itemize}
\item[RQ1] Which are the current software engineering practices for engineering robotic applications and which are their limitations?
\item[RQ2] Which software engineering practices can be developed in order to improve the process of engineering robotic applications?
\item[RQ3] Which are the applicable strategies to manage a heterogeneous robotic application with only partial knowledge of a dynamic environment?
\end{itemize}

\textbf{Contributions.} 
Our contributions are listed in the following:

\begin{enumerate}
\item Definition of a software architecture able to structure a robotic team;
%\item Validation of the architecture in a real-world scenario
\item Implementation of a software platform where all the algorithms and tools developed can be plugged in;
\item Definition of configuration mechanisms to enable \emph{start-up configuration} and \emph{run-time configuration};
\item Integration of an approach based on ROS+REST for the internal communication between robots;
\item Development of the algorithms in charge of managing the robotic team, based on:
\begin{enumerate}
\item Management of emergent properties;
\item Selection of collaborative adaptation techniques.
\end{enumerate}
\end{enumerate}

At this moment the software architecture and the communication mechanisms are already developed.
The software platform is being continuously developed and we intend to start dealing with the configuration facilities in the near future.
The development of the algorithms in charge of managing the robotic team are planned to be addressed during the second part of the project.

\textbf{Organization.} 
Section~\ref{sec:approach} describes the research approach of the current work and answers the first research question.
In Section ~\ref{sec:single} we present the process that we follow for engineering robotic applications and answer RQ2.
Then, in Section~\ref{sec:multi} we explain our plan for managing a robotic team while answering RQ3.
In Section~\ref{sec:related}, we introduce different works with a similar scope and position our research.
Section~\ref{sec:validation} explains our validation plan.
It concludes with Section~\ref{sec:conclusion} with final remarks.