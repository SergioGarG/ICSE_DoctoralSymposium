Service robots are increasingly involved in human lives. 
They are more and more used in environments such as houses, airports, hospitals, and offices for performing navigation, transportation, and manipulation tasks. 
The World Robotic Survey~\cite{wrs:online} estimated 35 million indoor service robots to be sold by 2018, accumulating a sales value of \$12 billion since 2015. 
%The global sales of household and personal robots is expected to grow by 23.5\% per year~\cite{sheng:online}. 
This increase is accompanied with a huge progress in robot technology. %, especially in image processing, planning, control, and collaboration. 
Software engineering is key to sustaining this new technology.

A robot typically performs specialized tasks; however, some tasks are highly complex and require a team of robots, whose capabilities %(e.g., perception, manipulation, and actuation) 
are coordinated and supervised. 
Such teams also need to adapt to changes, such as of the environment, of the desired tasks, or of the robot (e.g., hardware failures). 
These demands drive the complexity of robot control software relying on appropriate software architectures. 
To tackle this complexity, we need to rethink design processes~\cite{Lee2008} by properly managing system integration and raising the abstraction levels, addressing qualities such as evolvability, configurability, scalability, and dependability.

According to the Multi-Annual Roadmap For Robotic in Europe~\cite{roadmap} there are no mature system development processes for robotic applications, meaning that robotic applications are nowadays created from scratch without following systematic-engineering methods.
In this context, the aim of the Co4Robots project is to establish systematic engineering process to facilitate the development of the software for animating robotic systems through the creation of reusable robot building blocks with well-defined interfaces and properties. 

The research conducted during the present project will be split in two parts.
The first one defines the best practices for engineering software robotic applications.
During this period we study the current software engineering practices for both single and multi-robotic systems.
Furthermore, throughout this part of the research we develop the \emph{Software platform} that will be allocated within every robot of our applications.
%An instance of our \emph{Software platform} will be deployed on each robot.
It will integrate the \emph{Software architecture} of the whole system and  the \emph{Configuration facilities}, which provide the required tools for configuring our architecture.% both at design and run-time.

The second part of the research aims to support the \emph{choreography} of robotic applications.
It is considered future work that will be tackled once the platform is completely realised.
In this context, choreography means the way of representing and controlling the interactions between multiple services of a system in a decentralized way.
%The main expected outcome of this part is to perform 
Our goal is to perform the choreography of a deployed team of potentially heterogeneous robots in dynamic environments with the presence of human beings.
In order to do so, issues as \emph{Emergent properties} ~\cite{DeAngelis2016} and selecting the most suitable \emph{Collaborative adaptation}~\cite{Yan2013} techniques must be addressed.

\noindent \textbf{Research Questions.} 
We state the following research questions:
\begin{itemize}
\item[RQ1] What are the current software engineering practices for engineering robotic applications and what are their limitations?
\item[RQ2] What software engineering practices can improve the process of engineering robotic applications?
\item[RQ3] What are the applicable strategies to manage a heterogeneous robotic application with only partial knowledge of a dynamic environment?
\end{itemize}

\noindent \textbf{Contributions.} 
Our contributions are:

\begin{enumerate}
\item Definition of a software architecture able to structure a robotic team;
%\item Validation of the architecture in a real-world scenario
\item Implementation of an extensible software platform where all the developed algorithms and tools can be plugged in;
\item Definition of configuration mechanisms to enable \emph{start-up configuration} and \emph{run-time configuration};
\item Integration of an approach based on REST over ROS for the communication among robots;
\item Development of the algorithms in charge of managing the robotic team, based on management of emergent properties and selection of collaborative adaptation techniques.
%\end{enumerate}
\end{enumerate}

At this moment, the software architecture and the communication mechanisms are already developed.
The software platform is being continuously developed and we intend to start dealing with the configuration facilities in the near future.
The development of the algorithms in charge of managing the robotic team are planned to be addressed during the second part of the project.

\noindent \textbf{Validation.} 
Our work is involved in the Co4Robots\footnote{\url{http://www.co4robots.eu/}} European project.
Being part of this project allows us not only to formulate but also to validate our research questions in real-world scenarios.
In order to validate the code and artifacts developed we defined three different approaches than can be followed depending on the artifact that we want to obtain.
The first approach consists in getting feedback from stakeholders and practitioners; this has been done, for instance, through presentations of our work during meetings and consortiums.
The second approach relies on simulation tools that allow us to validate our artifacts before implementing them in real robots.
The third approach consists in validation with real robots in real-world scenarios.
A video of an experiment consisting in a real robot being conducted by a human in a real-world scenario is provided in\footnote{\url{dropboxfolder}}.
