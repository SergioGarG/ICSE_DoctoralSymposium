--General conclusions (maybe use the same from the last paper but removing the discussion about the results)--

This work presented  \toolName, a novel decentralized planner for partially known environments.
\toolName\ solves the decentralized planning problem when partial robot applications are analyzed.
The results show that the effectiveness of \toolName\ is triggered when the computed possible plans are actually executable in the real model of the robotic application.
Furthermore, in several cases, \toolName\ was able to achieve missions that could not be completed by classical planners.

Future work and research directions include
\begin{enumerate*}
\item the study of appropriate policies to select between definitive and possible plans,.
These policies may consider the likelihood of possible plans to be actually executable by the partial robot application,  e.g.,  the probability that a door is open.
\item the use of more efficient planners to speed up plan computation.
These may be based for example on symbolic techniques.
\end{enumerate*}
