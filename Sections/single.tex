After the research of the current state of the art we proceeded with RQ2, that will be addressed during the first part of the PhD as well.
In order to answer it while explaining our proposed process we divided the development of our framework into different tasks, that correspond with the items stated in Section~\ref{sec:introduction}.

\subsection{Software architecture}
\label{sec:softwarearch}

We propose our software architecture, Self-adaptive dEcentralised Robotic Architecture (SERA) for the framework.
As its name indicates, it supports a real-time decentralized robot coordination to accomplish missions with teams of robots. 
Furthermore, it is self-adaptive, responding to external and internal events by computing new strategies to achieve the desired goals.
SERA is inspired by and extends concepts of existing proposals for robot software architectures from the literature. 
Specifically, we inspected architectures identified by a mapping study of Ahmad et al.~\cite{Ahmad201616}, which investigated software architectures for robotics systems to identify and analyze the relevant literature based on 56 peer-reviewed papers.
The aforementioned architecture is three layers architecture that is strongly influenced by the well-known work of Kramer and Magee~\cite{kramer}.
Furthermore, it is a component-based type of architecture, so functionalities of the system are encapsulated in modules called ``components".
It is important to remark that the aim of our project is to build a system that can be easily used by not technical users, so we had to define a way for them to command the missions to the robotic team.
For this reason we added a central station that is just used during design-time in order to allocate a graphical interface to be used by a final user.

We defined SERA by first conceiving an architecture for a single robot. 
Then, we extended and refined it in order to iteratively extend and refine the architecture towards enabling communication among robots and collective adaptations. 
Thus, all the robots have a instantiation of the reference architecture but are also able to communicate and share information with the rest of the team, making possible the collective adaptation. 

SERA was already tested during an Integration Meeting of the project, where it demonstrate that can support the performance of a robot achieving different complex missions ---i.e. collaborative transportation with an human being, autonomous driving in a dynamic environment.

%\begin{figure*}[!t]
%\begin{center}
%\includegraphics[width=1\linewidth]{Figures/InstanceMultiRobot_Graffle.pdf}
%\caption{Software architecture.}
%\label{fig:arch}
%\end{center}
%\end{figure*}

\subsection{Software platform}

As explained before, the Software platform will integrate the Software architecture, all the tools and software created by developers and the Configuration facilities.
The platform is also a collection of the components that compose the architecture, a kind of library.
In our project the components of our architecture represent ROS~\cite{Quigley2009} nodes and packages, so the platform itself should be based on this middleware.
All this components are developed abstracting the communication problems since we rely on the interfaces defined in the architecture.
It not only significantly reduces the complexity of the code but also triggers the modularity of our system making possible exchanging the components based on the context.

We not only plan to control the performance of the system that is running in each robot but also its behaviour.
Thus, the usage of a high-level behavior engine, flexibly applicable to numerous systems and scenarios is mandatory.
FlexBe~\cite{Schillinger2016} not only provides a way of defining the behaviour of the robot in different scenarios (as of the study cases) by can also be used for defining the work flow.
It also provides a graphical interface that simplifies enormously these tasks.
FlexBe encapsulates functionalities of the robotic application, as our architecture does within components, and provides a way of orchestrate them so we keep the modularization of our system.
As with ROS, an instantiation of FlexBe will be deployed in every robot.
The integration of FlexBe is driven for the necessity of a software development methodology as MDE in our project.
It improves the modularity, variability and reusability of our system facilitating the development of the software for animating robotic systems through the creation of reusable robot building blocks with well-defined interfaces and properties.

Finally, in order to communicate each robot with its teammates we implemented an approach based on ROS+REST.
So, using a suitable component that works as an interface we are able to send messages in form of services between robots.
%In this way, each robot has an instance of ROS running in their own local environment so we can deploy a whole team of robots avoiding a central master node and the problems related with this approach (i.e. bottleneck issues, less robustness facing failures of a node, etc.), specially working with the ROS middleware.

\subsection{Configuration facilities}

Since the components of our architecture are exchangeable our next short-term goal is to define configuration facilities that can be applied to our system depicted in the architecture.
It will allow to our applications to support two things:

\begin{enumerate}
\item Being customizable at design-time, so we can configure its components based on the requirements of our context (i.e. hardware installed in each robot, environment where they will be deployed, etc.)
\item To self-adapt or self-configure at run time, so each robot can apply changes in its configuration based on emergent events of the environment or failures of their system.
\end{enumerate}

In order to do so we will implement pluginlib~\footnote{http://wiki.ros.org/pluginlib}, a package that uses the ROS build infrastructure and provides tools for writing and dynamically load plugins.



