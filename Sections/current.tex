Currently, our main focus lays on the RQ1, learning, defining and differencing between software engineering practices regarding single-robot and multi-robot systems.
Our software architecture, Self-adaptive dEcentralised Robotic Architecture (SERA) is already defined.
As its name indicates, it supports a real-time decentralized robot coordination to accomplish missions with teams of robots. 
Furthermore, it is self-adaptive, responding to different changes by computing new strategies to achieve the desired goals.
SERA was already tested during an Integration Meeting of the project, where it demonstrate that can support the performance of a robot achieving different complex missions ---i.e. collaborative transportation with an human being, autonomous driving in a dynamic environment.

The aforementioned architecture follows the component-based style, so the main robotic functionalities are encapsulated in different modules or "components".
All this components are developed abstracting the communication capabilities since we rely on the interfaces defined in the architecture.
It not only significantly reduces the complexity of the code but also triggers the modularity of our system making possible exchanging the components that conform our architecture.

Since the components of our architecture are exchangeable our next step is to define configuration facilities in order to allow the customization of the robotic application during design-time or its self-adaptation at run-time.
\sergio{cita a rosplug lib?}

Finally, the communication approach based on ROS+REST is already implemented, allowing the famous robotic middleware to share information between different robots in a decentralized way using services.

